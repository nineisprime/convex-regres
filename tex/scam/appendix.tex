\section{Appendix}
 
 
 \subsection{Proof of the Deterministic Condition for Sparsistency}
 \label{sec:deterministic_proof}
 
We restate Theorem~\ref{thm:deterministic} first for convenience. 
The following holds regardless of whether we impose the Lipschitz condition in optimization~\ref{opt:alternate_opt} and optimization~\ref{opt:alternate_opt_concave}.

 
\begin{theorem} 
Let $\{\hat{d}_k \}_{k \in S}$ be a minimizer of the restricted regression, that is, the solution to optimization (\ref{opt:alternate_opt}) where we restrict $k \in S$. 
Let $\hat{r} \coloneqq Y - \sum_{k \in S} \bar{\Delta}_k \hat{d}_k$ be the restricted regression residual. 

Suppose for all $k\in S^c$ and for all $i=1,...,n$, $\lambda_n > | \frac{1}{2n}
\hat{r}^\tran \mathbf{1}_{(i:n)}|$ where $\mathbf{1}_{(i:n)}$ is 1 on the coordinates of the $i$-th largest to the $n$-th largest entries of $X_k$ and 0 elsewhere.
Then the following are true:
\begin{enumerate}
\item Let $\hat{d}_k = 0$ for $k \in S^c$, then \{$\hat{d}_k\}_{k=1,...,p}$ is an optimal solution to optimization~\ref{opt:alternate_opt}. Furthermore, any solution to the optimization program \ref{opt:alternate_opt} must be zero on $S^c$.
\item For all $k \in S^c$, the solution to optimization~\ref{opt:alternate_opt_concave} must be zero and must be unique.
\end{enumerate}
\end{theorem}


\begin{proof} 
We will omit the Lipschitz constraint in our proof here. It can be
added in and easily verified that the result of the theorem still holds.

We begin by considering the first item in the conclusion of the theorem.
We will show that with $\{\hat{d}_k\}_{k=1,..,p}$ as constructed, we
can set the dual variables to satisfy the 
complementary slackness and stationary conditions: $\nabla_{d_k} \mathcal{L}(\hat{d})  = 0$ for all $k$. 

The Lagrangian is
\begin{equation}
\label{eqn:full_lagrange}
\mathcal{L}( \{ d_k \}, \nu) = 
  \frac{1}{2n} \Big\| 
    Y - \sum_{k=1}^p  \bar{\Delta}_k d_k  \Big\|_2^2 + 
    \lambda \sum_{k=1}^p \| \bar{\Delta}_k d_k \|_\infty -
    \sum_{k=1}^p \sum_{i=2}^{n-1} \nu_{ki} d_{ki} 
\end{equation}
with the constraint that $\nu_{ki} \geq 0$ for all $k,i$.

Because $\{\hat{d}_k\}_{k \in S}$ is by definition the optimal solution of the restricted regression, it is a consequence that stationarity holds for $k \in S$, that is, $\partial_{ \{ d_k \}_{k \in S} } \mathcal{L}(d) = 0$, and that the dual variables $\nu_k$ for $k \in S$ satisfy complementary slackness.

We now verify that stationarity holds also for $k \in S^c$. We fix one dimension $k \in S^c$ and let $\hat{r} = Y - \sum_{k' \in S} \bar{\Delta}_{k'} \hat{d}_{k'}$. 
The Lagrangian form of the optimization, in terms of just $d_k$, is
\[
\mathcal{L}(d_k, \nu_k) =
  \frac{1}{2n} \big\| Y - \sum_{k' \in S} \bar{\Delta}_{k'} d_{k'} 
  -  \bar{\Delta}_k d_k \big\|_2^2 
   + \lambda \| \bar{\Delta}_k d_k\|_\infty
  - \sum_{i=2}^{n-1} \nu_{ki} d_{ki}
\]
with the constraint that $v_i \geq 0$ for all $i$. 

The derivative of the Lagrangian is
\begin{align*}
\partial_{d_k} \mathcal{L}(d_k) =  -\frac{1}{n} \bar{\Delta}_k^\tran ( Y - \sum_{k'\in S} \bar{\Delta}_{k'} d_{k'}  - \bar{\Delta}_k d_k )
        + \lambda \bar{\Delta}_k^\tran \mathbf{u}
      - \nu_k
\end{align*}
where $\mathbf{u}$ is the subgradient of $\| \bar{\Delta}_k d_k \|_\infty$, an $n$-vector such that $\| \mathbf{u}_T \|_1 = 1$ where $T = \{ i \,:\,  (\bar{\Delta}_k d_k)_i = \| \bar{\Delta}_k d_k \|_\infty \}$.

We now substitute in $d_{k'} = \hat{d}_{k'}$ for $k' \in S$, $d_k = 0$ for $k \in S$, and $r = \hat{r}$ and show that the duals can be set in a way to ensure that the derivatives are equal to zero,
\begin{align*}
\partial_{d_k} \mathcal{L}(\hat{d}_k) = -\frac{1}{n} \bar{\Delta}_k^\tran\hat{r} + \lambda \bar{\Delta}_k^\tran \mathbf{u}
           - \nu_k = 0 
\end{align*}
where $\| \mathbf{u} \| \leq 1$ and $\nu_k \geq 0$. It clear that to show stationarity, we only need to show that $-\frac{1}{n} \bar{\Delta}_k^\tran \hat{r} + \lambda \bar{\Delta}_k^\tran \mathbf{u} \geq 0$ where the inequality is element-wise.

Let us reorder the samples so that the $i$-th sample is the $i$-th smallest sample. 
We will construct $\gamma = 0$, and $\mathbf{u} = (-a, 0, ..., a)$ for
some $0 < a < 1/2$, where the coordinates of $\mathbf{u}$ correspond
to the new sample ordering. 
We then just need to show that
\begin{align*}
- \frac{1}{n} \bar{\Delta}_k^\tran \hat{r} + \lambda \bar{\Delta}_k^\tran \mathbf{u} &\geq 0 \quad \Leftrightarrow \\
- \frac{1}{n} \Delta_k^\tran \hat{r} + \lambda \Delta_k^\tran \mathbf{u} &\geq 0 \quad \Leftrightarrow\\
- \frac{1}{n} \sum_{i > j} (X_{ki} - X_{kj}) \hat{r}_i + \lambda (X_{kn} - X_{kj})a &\geq 0 \quad 
   \textrm{for each } j \\
- \frac{1}{n} \sum_{i > j} \sum_{j < i' \leq i} \mathsf{gap}_{i'} \hat{r}_i 
    + \lambda (X_{kn} - X_{kj})a &\geq 0 \\
- \frac{1}{n} \sum_{i' > j} \mathsf{gap}_{i'} \sum_{i \geq i'} \hat{r}_i 
    + \lambda (X_{kn} - X_{kj})a &\geq 0 \\
- \frac{1}{n} \sum_{i' > j} \mathsf{gap}_{i'} \mathbf{1}_{(i':n)}^\tran \hat{r} 
   + \lambda (X_{kn} - X_{kj})a &\geq 0 
\end{align*}
where $\mathsf{gap}_i = X_{ki} - X_{k,i-1}$. If $\frac{1}{2n} |\mathbf{1}_{(i:n)}^\tran \hat{r}| \leq \lambda a$ for all $i=1,...,n$, then we have that
\begin{align*}
- \frac{1}{n} \sum_{i' > j} \mathsf{gap}_{i'} \mathbf{1}_{i':n}^\tran \hat{r}
   + \lambda (X_{kn} - X_{kj})a &\geq 0 \quad \Leftrightarrow \\
- \sum_{i' > j} \mathsf{gap}_{i'} \lambda a + \lambda (X_{kn} - X_{kj}) a &\geq 0\\
- (X_{kn} - X_{kj}) \lambda a + \lambda (X_{kn} - X_{kj}) a &\geq 0.
\end{align*}
We have thus proven that there exists one solution $\{ \hat{d}_k
\}_{k=1,...,p}$ such that $\hat{d}_k = 0$ for all $k \in
S^c$. Furthermore, we have shown that the subgradient variables
$\mathbf{u}_k$ of the solution $\{ \hat{d}_k \}$ can be chosen such
that $\| \mathbf{u}_k \|_1 < 1$ for all $k \in S^c$.  

We now prove that if $\{ \hat{d}'_k \}_{k = 1,..., p}$ is another
solution, then it must be that $\hat{d}'_k = 0$ for all $k \in S^c$ as
well.  We first claim that $\sum_{k=1}^p \bar{\Delta}_k \hat{d}_k =
\sum_{k=1}^p \bar{\Delta}_k \hat{d}'_k$. If this were not true, then a
convex combination of $\hat{d}_k, \hat{d}'_k$ would achieve a strictly
lower objective on the quadratic term. More precisely, let $\zeta \in
[0,1]$. If $\sum_{k=1}^p \bar{\Delta}_k \hat{d}'_k \neq \sum_{k=1}^p
\bar{\Delta}_k \hat{d}_k$, then $\| Y - \sum_{k=1}^p \bar{\Delta}_k
\big( \hat{d}_k + \zeta ( \hat{d}'_k - \hat{d}_k) \big) \|_2^2$ is
strongly convex as a function of $\nu$. Thus, it cannot be that
$\hat{d}_k$ and $\hat{d}'_k$ both achieve optimal objective, and we
have reached a contradiction.

Now, we look at the stationarity condition for both $\{ \hat{d}_k \}$
and $\{ \hat{d}'_k \}$. Let $\mathbf{u}_k \in \partial \|
\bar{\Delta}_k \hat{d}_k \|_\infty$ and let $\mathbf{u}'_k
\in \partial \| \bar{\Delta}_k \hat{d}'_k \|_\infty$ be the two sets
of subgradients. Let $\{ \nu_{ki} \}$ and $\{
\nu'_{ki} \}$ be the two sets of positivity dual
variables, for $k=1,..,p$ and $i=1,...n-1$.  Note that since there is no positivity constraint on
$d_{k1}$, we let $\nu_{k1} = 0$ always.

Let us define $\bar{\Delta}$, a $n \times p(n-1)$ matrix, to denote the column-wise concatenation of $\{ \bar{\Delta}_k \}_k$ and $\hat{d}$, a $p(n-1)$ dimensional vector, to denote the concatenation of $\{ \hat{d}_k \}_k$. With this notation, we can express $\sum_{k=1}^p \bar{\Delta}_k \hat{d}_k = \bar{\Delta} \hat{d}$.

Since both solutions $(\hat{d}, \mathbf{u}, \nu)$ and $(\hat{d}',
\mathbf{u}', \nu')$ must satisfy the stationarity condition, we have
that
\[
\bar{\Delta}^\tran ( Y - \bar{\Delta} \hat{d} ) 
   + \lambda \sum_{k=1}^p \bar{\Delta}_k^\tran \mathbf{u}_k - \nu = 
\bar{\Delta}^\tran ( Y - \bar{\Delta} \hat{d}' ) 
   + \lambda \sum_{k=1}^p \bar{\Delta}_k^\tran \mathbf{u}'_k - \nu' = 0.
\] 
Multiplying both sides of the above equation by $\hat{d}'$,
\[
\hat{d}'^{\tran}  \bar{\Delta}^\tran ( Y - \bar{\Delta} \hat{d} ) 
    + \lambda \sum_{k=1}^p \hat{d}'^\tran_k \bar{\Delta}_k^\tran \mathbf{u}_k - \hat{d}'^\tran \nu = \hat{d}'^{\tran}  \bar{\Delta}^\tran ( Y - \bar{\Delta} \hat{d}' ) 
    + \lambda \sum_{k=1}^p \hat{d}'^\tran_k \bar{\Delta}_k^\tran \mathbf{u}'_k - \hat{d}'^\tran \nu'.
\]
Since $\bar{\Delta} \hat{d}_k = \bar{\Delta} \hat{d}$, $\hat{d}'^\tran \nu' = 0$ (complementary slackness), and $\hat{d}'^\tran_k \bar{\Delta}_k^\tran \mathbf{u}'_k  = \| \hat{f}'_k \|_\infty$ (where $\hat{f}'_k = \bar{\Delta}_k \hat{d}'_k$), we have that
\[
\lambda \sum_{k=1}^p \hat{d}'^\tran_k \bar{\Delta}_k^\tran \mathbf{u}_k - \hat{d}'^\tran \nu = \lambda \sum_{k=1}^p \| \hat{f}'_k \|_\infty.
\]
On one hand, $\hat{d}'$ is a feasible solution so $\hat{d}'^\tran \nu \geq 0$ and so 
\[
\sum_{k=1}^p \hat{d}'^\tran_k \bar{\Delta}_k^\tran \mathbf{u}_k \geq \sum_{k=1}^p \| \hat{f}'_k \|_\infty .
\]
On the other hand, by H\"older's inequality,
\begin{align*}
\sum_{k=1}^p \hat{d}'^\tran_k \bar{\Delta}_k^\tran \mathbf{u}_k &\leq 
   \sum_{k=1}^p \| \hat{f}'_k \|_\infty \|\mathbf{u}_k \|_1 .
\end{align*}
Since $\mathbf{u}_k$ can be chosen so that $\| \mathbf{u}_k \|_1 < 1$ for all $k \in S^c$, we would get a contradiction if $\| \hat{f}'_k \|_\infty > 0$ for some $k \in S^c$. We thus conclude that $\hat{d}'$ must follow the same sparsity pattern.


The second item in the theorem concerning optimization~\ref{opt:alternate_opt_concave} is proven in exactly the same way. 
The Lagrangian of optimization~\ref{opt:alternate_opt_concave} is
\[
\mathcal{L}_{\trm{cave}}(d_k, \nu_k) = 
  \frac{1}{2n} \big\| \hat{r} - \bar{\Delta}_k d_k \big \|_2^2 + 
  \lambda \| \bar{\Delta}_k d_k \|_\infty + \sum_{k=1}^p \sum_{i=2}^{n-1} \nu_{ki} d_{ki}.
\]
The same reasoning applies to show that $\hat{d}_k = 0$ satisfies KKT conditions sufficient for optimality.
\end{proof}
 
 
 
 \subsection{Proof of False Positive Control}
 \label{sec:false_positive_proof}
 
 We note that in the following analysis the symbols $c,C$ represent
 absolute constants. We will often abuse notation and ``absorb'' new
 absolute constants into $c, C$; the actual value of $c, C$ could thus
 vary from line to line.
We first restate the theorem for convenience. 

\begin{theorem} 
Suppose assumptions A1-A4 hold. Suppose $\sqrt{\log 2np} \geq 1$, and define $\tilde{\sigma} \equiv \max(\sigma, B)$.

If $\lambda_n \geq 8 s \tilde{\sigma}  \sqrt{ \frac{1}{n} \log^2 np}$ then, with probability at least $ 1 - \frac{12}{n}$, for all $k \in S^c$, and for all $i'=1,...,n$
\[
\lambda_n > \Big| \frac{1}{2n}\hat{r}^\tran \mathbf{1}_{(i':n)_k} \Big|.
\]
Therefore, for all $k \in S^c$, both the AC solution $\hat{f}_k$ from optimization~\ref{opt:alternate_opt}, and the DC solution $\hat{g}_k$ from optimization~\ref{opt:alternate_opt_concave} are zero. 
\end{theorem}

\begin{proof}
The key is to note that $\hat{r}$ and $\Delta_{k,j}$ are independent for all $k \in S^c,j=1,...,n$ because $\hat{r}$ is only dependent on $X_{S}$.

Fix $j$ and $i$. Then $\hat{r}^\tran \mathbf{1}_{(i':n)_k}$ is the sum
of $n-i'+1$ random coordinates of $\hat{r}$. We will use
Serfling's theorem on the concentration of measure of sampling without
replacement (Corollary~\ref{cor:serfling}). We must first bound $\|
\hat{r} \|_\infty$ and $\frac{1}{n} \sum_{i=1}^n \hat{r}_i$ before we
can use Serfling's results however.

\vskip5pt
\textbf{Step 1}: {\it Bounding $\| \hat{r} \|_\infty$.} We have $\hat{r}_i = f_0(x_i) + w_i - \hat{f}(x_i)$ where
$\hat{f}(x_i) = \sum_{k \in S} \bar{\Delta}_k \hat{d}_k$ is the convex
additive function outputted by the restricted regression. Note that
both $f_0(x_i)$ and $\hat{f}(x_i)$ are bounded by $sB$. 
Because $w_i$ is sub-Gaussian, $|w_i| \leq  \sigma \sqrt{2\log \frac{2}{\delta}}$ with probability at most $1-\delta$. By union bound across $i=1,...,n$, we have that $\| w\|_\infty \leq \sigma \sqrt{ 2 \log \frac{2n}{\delta}}$ with probability at most $1 - \delta$.

Putting these observations together,
%and take another union bound across all $j$ and all $i'$:
\begin{align}
\| \hat{r} \|_\infty &\leq 2sB + \sigma \sqrt{ 2\log \frac{2n}{\delta}}) \nonumber \\
      &\leq 4 s \tilde{\sigma} \sqrt{\log \frac{2n}{\delta}} \label{eqn:stepone_rhat}
\end{align}
with probability at least $1 - \delta$, where we have defined
$\tilde{\sigma} = \max(\sigma, B)$,
and assumed that $\sqrt{\log \frac{2np}{\delta}} \geq 1$, which is acceptable since $\delta \leq 1$ and the theorem presupposes that $\sqrt{ \log 2np } \geq 1$.

\vskip5pt
\textbf{Step 2}: {\it Bounding $| \frac{1}{n} \hat{r}^\tran \mathbf{1}
  |$.}  We have that 
\begin{align*}
\frac{1}{n} \hat{r}^\tran \mathbf{1} &= 
    \frac{1}{n} \sum_{i=1}^n f_0(x_i) + w_i - \hat{f}(x_i) \\
  &= \frac{1}{n} \sum_{i=1}^n f_0(x_i) + w_i \quad \trm{ ($\hat{f}$ is centered)}.
\end{align*}
Since $|f_0(x_i)| \leq sB$, the first term $| \frac{1}{n} \sum_{i=1}^n
f_0(x_i)|$ is at most $sB \sqrt{\frac{2}{n} \log \frac{2}{\delta}}$
with probability at most $1-\delta$ by Hoeffding's inequality. Since
$w_i$ is sub-Gaussian, the second term $|\frac{1}{n} \sum_{i=1}^n
w_i|$ is at most $\sigma \sqrt{ \frac{2}{n} \log \frac{2}{\delta}}$
with probability at most $1-\delta$.  
Taking a union bound, we have that 
\begin{align}
| \frac{1}{n} \hat{r}^\tran \mathbf{1}| &\leq sB \sqrt{\frac{2}{n} \log \frac{4}{\delta}} +  \sigma \sqrt{\frac{2}{n} \log \frac{4}{\delta}} \nonumber \\
  &\leq 4 s \tilde{\sigma} \sqrt{\frac{1}{n} \log \frac{4}{\delta}} \label{eqn:steptwo_rhat}
\end{align}
with probability at least $1-\delta$.

\vskip5pt
\textbf{Step 3}: {\it Apply Serfling's theorem.}  
For any $k \in S^c$, Serfling's theorem states that with probability at least $1 - \delta$
\begin{align*}
\Big
|\frac{1}{n} \hat{r}^\tran \mathbf{1}_{(i':n)_k}\Big| \leq
   2\| \hat{r} \|_\infty \sqrt{ \frac{1}{n} \log \frac{2}{\delta}} + 
   \Big|\frac{1}{n} \hat{r}^\tran \mathbf{1} \Big|
\end{align*}
We need Serfling's theorem to hold for all $k = 1,...,p$ and $i' =
1,...,n$. We also need the events that $\|\hat{r}\|_\infty$ and $|
\frac{1}{n} \hat{r}^\tran \mathbf{1}|$ are small to hold. Using a
union bound, with probability at least $1-\delta$, for all $k,i'$,
\begin{align*}
\Big
|\frac{1}{n} \hat{r}^\tran \mathbf{1}_{(i':n)_k}\Big| &\leq
   2\| \hat{r} \|_\infty \sqrt{ \frac{1}{n} \log \frac{6np}{\delta}} + 
   \Big|\frac{1}{n} \hat{r}^\tran \mathbf{1} \Big|\\
  &\leq 4s\tilde{\sigma}\sqrt{\log \frac{6n}{\delta}} \sqrt{\frac{1}{n}\log \frac{6np}{\delta}} + 4 s \tilde{\sigma} \sqrt{\frac{1}{n}\log \frac{12}{\delta}} \\
  &\leq 8 s\tilde{\sigma} \sqrt{ \frac{1}{n} \log^2 \frac{12np}{\delta}}
\end{align*}
In the second inequality, we used equation~\eqref{eqn:stepone_rhat}
and equation~\eqref{eqn:steptwo_rhat} from steps 1 and 2
respectively. Setting $\delta = \frac{12}{n}$ gives the desired
result.
\end{proof}

 
%%%%%%%%%%%%%%%%%%%%%%%%%%%%%%
%%
%%
%%
%%
%%%%%%%%%%%%%%%%%%%%%%%%%%%%%% 
 
 \subsection{Proof of False Negative Control}
 \label{sec:false_negative_proof}
 We begin by introducing some notation.
 \subsubsection{Notation} 
\label{sec:false_negative_proof_notations}
If $f : \mathbb{R}^s \rightarrow \R$, we define $\| f \|_P \equiv \E f(X)^2$. 
Given samples $X_1,...,X_n$, we denote $\| f \|_n \equiv \frac{1}{n} \sum_{i=1}^n f(X_i)^2$ and $\langle f, g \rangle_n \equiv \frac{1}{n} \sum_{i=1}^n f(X_i) g(X_i)$. 

Let $\mathcal{C}^1$ denote the set of univariate convex functions supported on $[-b,b]$. Let $\mathcal{C}^1_B \equiv \{ f \in \mathcal{C}^1 \,:\, \| f \|_\infty \leq B \}$ denote the set of $B$-bounded univariate convex functions. 
Define $\mathcal{C}^s$ as the set of convex additive functions and
$\mathcal{C}^s_B$ likewise as the set of convex additive functions
whose components are $B$-bounded:
\begin{align*}
\mathcal{C}^s &\equiv \{ f \,:\, f = \sum_{k=1}^s f_k, \,
   f_k \in \mathcal{C}^1 \} \\
\mathcal{C}^s_B &\equiv \{ f \in \mathcal{C}^s \,:\, 
f = \sum_{k=1}^s f_k, \, \| f_k \|_\infty \leq B \}.
\end{align*}
Let $f^*(x) = \sum_{k=1}^s f^*_k(x_k)$ be the population risk minimizer:
\[
f^* = \arg\min_{f \in \mathcal{C}^s} \| f_0 - f^* \|_P^2
\]
We let $sB$ be an upper bound on $\| f_0 \|_\infty$ and $B$ be an
upper bound on $\| f^*_k \|_\infty$. 
It follows that $\|f^* \|_\infty \leq s B$.

We define $\hat{f}$ as the empirical risk minimizer:
\[
\hat{f} = \arg\min \Big \{ \| y - f \|_n^2 + \lambda \sum_{k=1}^s \| f_k \|_\infty 
    \,:\, f \in \mathcal{C}^s_B,\, \mathbf{1}_n^\tran f_k = 0 \Big \}
\]
For $k \in \{1,...,s\}$, define $g^*_k$ to be the decoupled concave population risk minimizer
\[
g^*_k \equiv \argmin_{g_k \in \mh \mathcal{C}^1} \| f_0 - f^* - g_k \|_P^2 .
\]
In our proof, we will analyze $g^*_k$ for each $k$ such that $f^*_k = 0$. Likewise, we define the empirical version:
\[
\hat{g}_k \equiv \argmin \Big\{ \| f_0 - \hat{f} - g_k \|_n^2 \,:\, g_k \in \mh \mathcal{C}^1_B \,, \mathbf{1}_n^\tran g_k = 0 \Big\}.
\]
By the definition of the AC/DC procedure, $\hat{g}_k$ is defined only
for an index $k$ that has zero as the convex additive approximation.


\subsubsection{Proof}
 
By additive faithfulness of the AC/DC procedure, it is known that $f^*_k \neq 0$ or $g^*_k \neq 0$ for all $k \in S$. 
Our argument will be to show that the risk of the AC/DC estimators $\hat{f}, \hat{g}$ tends to the risk of the population optimal functions $f^*, g^*$:
\begin{align*}
\| f_0 - \hat{f} \|^2_P & = \| f_0 - f^* \|^2_P + \trm{err}_+(n) \\
\| f_0 - f^* - \hat{g}_k \|^2_P &=  \| f_0 - f^* - g^*_k \|^2_P + \trm{err}_-(n) 
       \quad \trm{for all $k \in S$ where $f^*_k = 0$},
\end{align*}
where the estimation errors $\trm{err}_+(n)$ and $\trm{err}_-(n)$ decrease with $n$ at some rate. 

Assuming this, suppose that $\hat{f}_k = 0$ and $f^*_k \neq 0$. Then when $n$ is large
enough such that $\trm{err}_+(n)$ and $\trm{err}_-(n)$ are smaller
than $\alpha_+$ and $\alpha_-$ defined in
equation~\eqref{eqn:signal_level_defn}, we reach a contradiction.
This is because the risk $\| f_0 - f^* \|_P$ of $f^*$ is
strictly larger by $\alpha_+$ than the risk of the best approximation
whose $k$-th component is constrained to be zero.
Similarly, suppose $f^*_k = 0$ and $g^*_k \neq 0$. Then when $n$ is large
enough, $\hat{g}_k$ must not be zero.

Theorem~\ref{thm:convex_consistent} and
Theorem~\ref{thm:concave_consistent} characterize $\trm{err}_+(n)$ and
$\trm{err}_-(n)$ respectively.

\begin{theorem}
\label{thm:convex_consistent}
Let $\tilde{\sigma} \equiv \max(\sigma, B)$, and let $\hat{f}$ be the
minimizer of the restricted regression with $\lambda \leq 9 s
\tilde{\sigma} \sqrt{ \frac{1}{n} \log^2 np}$.
Suppose $n \geq c_1 \max(\sqrt{sB}, B)$.
Then with probability at least $1-\frac{C}{n}$,
\begin{align}
\|f_0 - \hat{f} \|_P^2 - \| f_0 - f^* \|_P^2 
&\leq c B^2 \tilde{\sigma} \sqrt{ \frac{s^5}{n^{4/5}} \log^2 np},
\end{align}
where $c, C$ are constants, possibly dependent on $c_1, b$.
\end{theorem}


\begin{proof}
Our proof proceeds in three steps.  First, we bound
the difference of empirical risks $\|f_0 - \hat{f} \|_n^2 - \| f_0 -
f^* \|_n^2$.  Second, we bound the cross-term in the bound using
a bracketing entropy argument for convex function classes.  Finally, 
we combine the previous two steps to complete the argument.

\textbf{Step 1.} The function $\hat{f}$ minimizes the penalized
empricial risk by definition. We would thus like to say that the
penalized empirical risk of $\hat{f}$ is no larger than that of
$f^*$. We cannot do a direct comparison, however, because the empirical mean
$\frac{1}{n} \sum_i f^*_k(x_{ik})$ is close to, but not exactly
zero. We thus have to work first with the function $f^* - \bar{f}^*$.
We have that
\begin{align*}
\| y - \hat{f} \|_n^2 + \lambda \sum_{k=1}^s \| \hat{f}_k \|_\infty &\leq
  \| y - f^* + \bar{f}^* \|_n^2 + \lambda \sum_{k=1}^s \| f^*_k - \bar{f}^*_k \|_\infty 
\end{align*}
Plugging in $y = f_0 + w$, we obtain
\begin{align*}
\| f_0 + w - \hat{f} \|_n^2 + \lambda \sum_{k=1}^s \Big( \| \hat{f}_k \|_\infty - 
    \| f^*_k - \bar{f}^*_k \|_\infty \Big) &\leq \|f_0 + w - f^* + \bar{f}^* \|_n^2 \\
\| f_0 - \hat{f} \|_n^2 + 2\langle w, f_0 - \hat{f} \rangle_n 
     +\lambda \sum_{k=1}^s \Big( \| \hat{f}_k \|_\infty - \|f^*_k -\bar{f}^*_k\|_\infty \Big) 
    &\\
\leq \| f_0 - f^* + \bar{f}^* \|_n^2 + 
    2 \langle w, f_0 - f^* + \bar{f}^* \rangle \\
\|f_0 - \hat{f} \|_n^2 - \| f_0 - f^* + \bar{f}^* \|_n^2 + 
    \lambda \sum_{k=1}^s \Big( \| \hat{f}_k \|_\infty - 
 \| f^*_k - \bar{f}^*_k \|_\infty \Big) &\leq 2 \langle w, \hat{f} - f^* + \bar{f}^* \rangle.
\end{align*}
The middle term can be bounded under the assumption that $\|f^*_k -
\bar{f}^*_k \|_\infty \leq 2B$; thus,
\begin{align*}
\|f_0 - \hat{f} \|_n^2 - \| f_0 - f^* + \bar{f}^* \|_n^2 
   &\leq 2 \langle w, \hat{f} - f^* + \bar{f}^* \rangle + \lambda 2 s B .
\end{align*}
Using Lemma~\ref{lem:remove_centering}, we can remove $\bar{f}^*$ from
the lefthand side. Thus with probability at least $1 - \delta$,
\begin{align}
\label{eqn:first_step_inequality}
\|f_0 - \hat{f} \|_n^2 - \| f_0 - f^* \|_n^2 
   &\leq 2 \langle w, \hat{f} - f^* + \bar{f}^* \rangle + \lambda 2 s B + c(sB)^2 \frac{1}{n} \log \frac{2}{\delta}.
\end{align}

%[TOOD: at this point, we still have to choose a value for $\lambda$]\\

%%%%%%%%%%%%%%%%%%%%%%%%%%%%%%
%% 
%% Step 2
%%
%%
%%%%%%%%%%%%%%%%%%%%%%%%%%%%%%

\textbf{Step 2.} We now upper bound the cross term $2 \langle w,
\hat{f} - f^* + \bar{f}^* \rangle$ using bracketing entropy.

Define $\mathcal{G} =\{ f - f^* + \bar{f}^* \,:\, f \in
\mathcal{C}^s_B \}$ 
as the set of convex additive functions centered around the function $f^* - \bar{f}^*$. 
By Corollary~\ref{prop:convexbracket_lp}, there is an $\epsilon$-bracketing of $\mathcal{G}$ whose size is bounded by $\log N_{[]}( 2\epsilon, \mathcal{G}, L_1(P)) \leq sK^{**} \left( \frac{2sbB}{\epsilon} \right)^{1/2}$, for all $\epsilon \in (0, sB \epsilon_3]$.
Let us suppose condition~\ref{cond:simplify_covering_number} holds. Then, by Corollary~\ref{cor:convexbracket_ln}, with probability at least $1-\delta$, each bracketing pair $(h_U, h_L)$ is close in $L_1(P_n)$ norm, i.e., for all $(h_U, h_L)$, 
$\frac{1}{n} \sum_{i=1}^n | h_U(X_i) - h_L(X_i) | \leq 2 \epsilon + sB \sqrt{ \frac{sK^{**}(2sbB
)^{1/2} \log \frac{1}{\delta}}{2\epsilon^{1/2} n}}$. We verify at the
end of the proof that 
condition~\ref{cond:simplify_covering_number} indeed holds.

For each $h \in \mathcal{G}$, there exists a pair $(h_U, h_L)$ such that $h_U(X_i) - h_L(X_i) \geq h(X_i) - h_L(X_i) \geq 0$. Therefore, with probability at least $1-\delta$, uniformly for all $h \in \mathcal{G}$:
$$
\frac{1}{n} \sum_{i=1}^n |h(X_i) - h_L(X_i)| \leq \frac{1}{n} \sum_{i=1}^n | h_U(X_i) - h_L(X_i)| \leq 2\epsilon +  (sB) \sqrt{ \frac{sK^{**}(2sbB)^{1/2} \log \frac{1}{\delta}}{2\epsilon^{1/2} n}}.
$$
We denote $\epsilon_{n,\delta} \equiv (sB) \sqrt{
  \frac{sK^{**}(2sbB)^{1/2} \log \frac{1}{\delta}}{2\epsilon^{1/2}
    n}}$. Let $\mathcal{E}_{[\,]}$ denote the event that for each $h
\in \mathcal{G}$, there exists $h_L$ in the $\epsilon$-bracketing such
that $\|h-h_L\|_{L_{P_n}} \leq 2\epsilon + \epsilon_{n, \delta}$. Then
$\mathcal{E}_{[\,]}$ has probability at most $1-\delta$ as shown.

Let $\mathcal{E}_{\|w\|_\infty}$ denote the event that $\| w \|_\infty
\leq \sigma \sqrt{ 2\log \frac{2n}{\delta}}$.  Then
$\mathcal{E}_{\|w\|_\infty}$ has probability at most $1-\delta$. We
now take an union bound over $\mathcal{E}_{\|w\|_\infty}$ and
$\mathcal{E}_{[\,]}$ and get that, with probability at most
$1-2\delta$, for all $h$
\[
|\langle w, h - h_L\rangle_n| \leq \| w \|_\infty \frac{1}{n} \sum_{i=1}^n |h(X_i) - h_L(X_i)| \leq
  \sigma \sqrt{2 \log \frac{4n}{\delta}} \left( \epsilon + \epsilon_{n,2\delta} \right).
\]
Because $w$ is a sub-Gaussian random variable, we have that for any fixed vector
$h_L = (h_L(X_1),...,h_L(X_n))$, with probability at least $1-\delta$,
$|\langle w, h_L \rangle_n | \leq \| h_L \|_n \sigma \sqrt{
  \frac{2}{n} \log \frac{2}{\delta} }$. Using another union bound, we
have that the event $\sup_{h_L} |\langle w, h_L \rangle| \leq sB
\sigma \sqrt{ \frac{2}{n}\log \frac{2 N_{[]}}{\delta}}$ has
probability at most $1-\delta$.

Putting this together, we have that
\begin{align*}
\lefteqn{|\langle w, h \rangle_n | \leq | \langle w, h_L\rangle_n| + |\langle w, h - h_L\rangle_n|}\\
\lefteqn{|\sup_{h \in \mathcal{G}} \langle w, h \rangle_n| \leq 
     | \sup_{h^L} \langle w, h_L \rangle_n | + \sigma \sqrt{2 \log \frac{2n}{\delta}} (2\epsilon + \epsilon_{n, 2\delta})} \\
   &\leq   sB \sigma \sqrt{ 2 \frac{ \log N_{[]} + \log \frac{1}{\delta}}{n}} + \sigma \sqrt{2 \log \frac{2n}{\delta}} (2\epsilon + \epsilon_{n, \delta}) \\
   &\leq  sB \sigma \sqrt{ 2 \frac{sK^{**} (2sbB)^{1/2} \log \frac{1}{\delta}}{n \epsilon^{1/2}}} +
   \sigma \sqrt{ 2\log \frac{2n}{\delta}} (2\epsilon + \epsilon_{n, \delta}) \\
   &\leq sB \sigma \sqrt{ 2\frac{sK^{**} (2sbB)^{1/2} \log \frac{1}{\delta}}{n \epsilon^{1/2}}} +
   2\sigma\sqrt{2 \log \frac{2n}{\delta}} \epsilon + sB \sigma \sqrt{2 \frac{sK^{**} (2sbB)^{1/2}\log \frac{1}{\delta}}{n \epsilon^{1/2}} \log \frac{2n}{\delta}} \\
   &\leq 2\sigma\sqrt{2\log \frac{2n}{\delta}} \epsilon + 2 sB \sigma \sqrt{ \frac{sK^{**} (2sbB)^{1/2} \log^2 \frac{2n}{\delta}}{n \epsilon^{1/2}}}.
\end{align*}
We require that $n$ is large enough so that $\epsilon \in (0, sB
\epsilon_3]$, and to ensure that
conditions~\eqref{cond:simplify_covering_number} hold, we need that
$\log n \geq 2$ and $n \geq c B$ where $c$ is a constant dependent
only on $K^{**}$ and $b$.
The conditions are satisfied if we set $\epsilon = sB \sqrt{ \frac{(s
    K^{**} (sBb)^{1/2})^{4/5}}{n^{4/5}} }$ and assume 
$n \geq c_1 \max( \sqrt{sB}, B)$, as stated in the theorem. 
%% ?? Setting $\epsilon$ in this way thus also balances the two terms.

In summary, we have that probability at least $1-\delta$,
\[
|\sup_{h \in \mathcal{G}} \langle w, h \rangle | \leq c sB \sigma \sqrt{ 
   \frac{s b^{1/2} \log^2 \frac{Cn}{\delta}}{n^{4/5}}}
\]
where we absorbed $K^{**}$ into the constant $c$ and the union bound multipliers into the constant $C$.
% Then, $\| h \|_n \leq \| h \|_\infty \leq 4sLb$ for all $h \in \mathcal{G}$.\\
% According to theorem~\ref{thm:chaining}, for all $\epsilon > \frac{1}{\sqrt{n}}\sigma c \int_0^R \sqrt{ \log N_2(t, \mathcal{G}) }dt \vee R$,
% \[
% P\Big( \sup_{h \in \mathcal{G}} \langle w, h \rangle_n \geq \epsilon \Big) \leq
%   4 \exp \Big( - \frac{ n \epsilon^2}{ c R^2 \sigma^2} \Big)
% \]
% where $R = 4sLb$ for our purpose.

% Restated, we have that, with probablity at least $1-\delta$,
% \[
% \sup_{h \in \mathcal{G}} | \langle w, h \rangle_n | \leq 
%    c R \sigma \sqrt{ \frac{1}{n} \log \frac{4}{\delta}} + 
%       \Big( \int_0^R \sqrt{\log N_2(t, \mathcal{G})}dt \vee R \Big)\, 
%        c \sigma \sqrt{\frac{1}{n}} 
% \]
% Now we evaluate the integral. Since $N_{\|\cdot\|_n}(t, \mathcal{G}) \leq N_\infty(t, \mathcal{G})$, we know that $\sqrt{\log N_{\|\cdot\|_n}(t, \mathcal{G})} \leq \sqrt{C s^{1.5} b L} t^{-1/4}$.
% \begin{align*}
% \int_0^R \sqrt{\log N_{\|\cdot\|_n}(t, \mathcal{G})} dt &\leq 
%       \sqrt{C s^{1.5} b L} \int_0^R t^{-1/4} dt \\ 
%  &= \sqrt{C s^{1.5} b L} \, \frac{4}{3} R^{3/4} \\
%  &= \sqrt{C s^{1.5} b L} \, c (sLb)^{3/4} \\
%  &\leq c (s b L)^2
% \end{align*}

% Coming back, we have, with probability at least $1-\delta$,
% \begin{align*}
% \sup_{h \in \mathcal{G}} | \langle w, h \rangle | &\leq 
%    c sLb \sigma \sqrt{ \frac{1}{n} \log \frac{4}{\delta} } + 
%     c (sLb)^2 \sigma \sqrt{ \frac{1}{n} } \\
%  &\leq c (sLb)^2\sigma \sqrt{ \frac{1}{n} \log \frac{4}{\delta} }
% \end{align*}
Plugging this result into equation~\eqref{eqn:first_step_inequality}
we get that, with probability at least $1 - 2\delta$,
\begin{align}
\|f_0 - \hat{f} \|_n^2 - \| f_0 - f^* \|_n^2 
   &\leq c sB \sigma \sqrt{ 
   \frac{s b^{1/2} \log^2 \frac{Cn}{\delta}}{n^{4/5}}}
   + \lambda 2 s B + c (sB)^2 \frac{1}{n} \log \frac{2}{\delta} \nonumber\\
\|f_0 - \hat{f} \|_n^2 - \| f_0 - f^* \|_n^2 
   &\leq c sB \sigma \sqrt{ 
   \frac{s b^{1/2} \log^2 \frac{Cn}{\delta}}{n^{4/5}}}
   + \lambda 2 s B \nonumber\\   
   &\leq c B \sigma 
    \sqrt{ \frac{s^3 b^{1/2}}{n^{4/5}} \log^2 \frac{Cn}{\delta}} + \lambda 2 sB
\label{eqn:second_step_inequality}
\end{align}

%%%%%%%%%%%%%%%%%%%%%%%%%%%%%%
%%
%% Step 3.
%%
%%
%%%%%%%%%%%%%%%%%%%%%%%%%%%%%%

\textbf{Step 3.} Continuing from
equation~\eqref{eqn:second_step_inequality}, we use
Lemma~\ref{lem:uniform_convergence} and another union bound to obtain
that, with probability at least $1-3\delta$,
\begin{align}
\|f_0 - \hat{f} \|_P^2 - \| f_0 - f^* \|_P^2 
   &\leq cB^2 \sigma 
    \sqrt{ \frac{s^3 b^{2/5}}{n^{4/5}} \log^2 \frac{Cn}{\delta}}
 +\lambda 2 s B + c B^3 \sqrt{ \frac{s^5}{n^{4/5}} \log \frac{2}{\delta}}
    \nonumber \\
&\leq c B^2 \tilde{\sigma} \sqrt{ \frac{s^5 b^{2/5}}{n^{4/5}} \log^2 \frac{Cn}{\delta}} + \lambda 2 sB \nonumber
\end{align}
Substituting in $\lambda \leq 9 s \tilde{\sigma} \sqrt{\frac{1}{n}
  \log^2 np}$ and $\delta = \frac{C}{n}$ we obtain the statement of
the theorem.
\end{proof}
 
%% End of "No False Negative" proof for the f's






\begin{theorem}
\label{thm:concave_consistent}
Let $\hat{g}_k$ denote the minimizer of the concave postprocessing
step with $\lambda_n \leq 9 sB \sqrt{\frac{1}{n} \log^2 np}$. Let $\tilde{\sigma} \equiv \max(\sigma, B)$.
Suppose $n$ is sufficiently large that $\frac{n^{4/5}}{\log^2 np} \geq B^4 \tilde{\sigma}^2 s^5$.
Then with probability at least $1- \frac{C}{n}$, for all $k=1,...,s$,
\[
\| f_0 - f^* - \hat{g}_k \|_P^2 - \| f_0 - f^* - g^*_k \|_P^2 \leq  c B^2 \sigma^{1/2} \sqrt[4]{ \frac{s^5}{n^{4/5}} \log^2 np}.
\]
\end{theorem}

\begin{proof}

Because we fit $\hat{g}_k$ against $f_0 - \hat{f}$ instead of $f_0 - f^*$, we need the following decomposition:
\begin{align}
\| f_0 - f^* - \hat{g}_k \|_P^2 - \| f_0 - f^* - g^*_k \|_P^2 = & \underbrace{\| f_0 - \hat{f} - \hat{g}_k \|_P^2 - \| f_0 - \hat{f} - g^*_k \|_P^2}_{\trm{term 1}} + \nonumber \\
   & \underbrace{\| f_0 - f^* - \hat{g}_k \|_P^2 - \| f_0 - \hat{f} - \hat{g}_k \|_P^2}_{\trm{term 2}} + \nonumber \\
   & \underbrace{\| f_0 - \hat{f} - g^*_k \|_P^2 - \| f_0 - f^* - g^*_k \|_P^2}_{\trm{term 3}}  \label{eqn:concave_init_decomposition}.
\end{align}
We now bound each of the terms. The proof proceeds almost identically to that of Theorem~\ref{thm:convex_consistent}, because convex and concave functions have the same bracketing number.

\textbf{Step 1.} To bound term 2, we start from the definition of
$\hat{g}_k$ and obtain
\begin{align*}
\| y - \hat{f} - \hat{g}_k \|_n^2 + \lambda_n \| \hat{g} \|_\infty &\leq
   \| y - \hat{f} - g^*_k \|_n^2 + \lambda_n \| g^* \|_\infty \\
\| y - \hat{f} - \hat{g}_k \|_n^2 &\leq \| y - \hat{f} - g^*_k \|_n^2 + \lambda_n 2B \\[10pt]
\| f_0 - \hat{f} - \hat{g}_k + w\|_n^2 & \leq \| f_0 - \hat{f} - g^*_k + w \|_n^2 
   +\lambda_n 2 B \\
\| f_0 - \hat{f} - \hat{g}_k \|_n^2 - \|f_0 -\hat{f} - g^*_k\|_n^2 &\leq
   2 \langle w, \hat{g}_k - g^*_k \rangle_n + \lambda_n 2B.
\end{align*}
Using the same analysis as in Step 2 of the proof of Theorem~\ref{thm:convex_consistent} but setting $s=1$, we have, with probability at least $1-\delta$,
\begin{align*}
\| f_0 - \hat{f} - \hat{g}_k \|_n^2 - \|f_0 - \hat{f} - g^*_k \|_n^2 &\leq
  c B^2 \sigma \sqrt{ \frac{b^{1/2}}{n^{4/5}} \log \frac{C}{\delta} }+ \lambda_n 2 B.
\end{align*}
Using the uniform convergence result of Lemma~\ref{lem:uniform_convergence}, with probability at least $1-\delta$,
\begin{align*}
\| f_0 - \hat{f} - \hat{g}_k \|_P^2 - \|f_0 - \hat{f} - g^*_k \|_P^2 &\leq
  c B^2 \sigma \sqrt{ \frac{1}{n} \log \frac{Cn}{\delta} }+ \lambda_n 2 B +
  c B^3 \sqrt{\frac{s^5}{n^{4/5}} \log \frac{2}{\delta} } \\
 &\leq c B^2 \tilde{\sigma} \sqrt{\frac{s^5}{n^{4/5}} \log \frac{C}{\delta}}+ \lambda_n 2B
\end{align*}
Finally, plugging in $\lambda_n \leq 9 s \tilde{\sigma} \sqrt{
  \frac{1}{n} \log^2 np}$, we obtain
\begin{align*}
\| f_0 - \hat{f} - \hat{g}_k \|_P^2 - \|f_0 - \hat{f} - g^*_k \|_P^2 &
\leq c B^2 \tilde{\sigma} \sqrt{\frac{s^5}{n^{4/5}} \log \frac{C}{\delta}}+ 
    2s B \tilde{\sigma} \sqrt{\frac{1}{n} \log^2 np}\\
\| f_0 - \hat{f} - \hat{g}_k \|_P^2 - \|f_0 - \hat{f} - g^*_k \|_P^2 &
\leq c B^2 \tilde{\sigma} \sqrt{\frac{s^5}{n^{4/5}} \log^2 \frac{Cnp}{\delta}}
\end{align*}
with probability at least $1-\delta$.

\textbf{Step 2.} To bound terms 2 and 3 we derive
\begin{align*}
\| f_0 - \hat{f} - g^*_k \|_P^2 - \| f_0 - f^* - g^*_k\|_P^2 &\leq 
    \| f_0 - \hat{f} \|_P^2 - \|f_0 - f^*\|_P^2 - 2\langle f_0 - \hat{f}, g^*_k \rangle_P
   + 2 \langle f_0 - f^*, g^*_k \rangle_P \\
 &\leq c B^2 \tilde{\sigma} \sqrt{ \frac{s^5}{n^{4/5}} \log^2 np} + 
    2 | \langle \hat{f} - f^*, g^*_k \rangle_P |  \\
 &\leq  c B^2 \tilde{\sigma} \sqrt{ \frac{s^5}{n^{4/5}} \log^2 np} +
    2 \| \hat{f} - f^* \|_P \| g^*_k \|_P \\
&\leq  c B^2 \tilde{\sigma} \sqrt{ \frac{s^5}{n^{4/5}} \log^2 np} +
   c B \sqrt{B^2 \tilde{\sigma} \sqrt{ 
                   \frac{s^5}{n^{4/5}} \log^2 np} }\\
&\leq  cB^2 \tilde{\sigma}^{1/2} \sqrt[4]{ 
                   \frac{s^5}{n^{4/5}} \log^2 np} 
\end{align*}
with probability at least $1-\frac{C}{n}$, by
Theorem~\ref{thm:convex_consistent}. To obtain the fourth inequality,
we used the fact that $\| \hat{f} - f^* \|^2 \leq \| f_0 - \hat{f}
\|_P^2 - \|f_0 - f^*\|_P^2$, which follows from the fact that $f^*$ is the
projection of $f_0$ onto the set of additive convex functions and the
set of additive convex functions is convex itself.
The last inequality holds due to the condition of the theorem that $n$
is large enough so that $B^2 \tilde{\sigma} \sqrt{ \frac{s^5}{n^{4/5}} \log^2 np} \leq 1$.
The same bound likewise holds for term 2.

\textbf{Step 3.} Collecting the results and plugging them into equation~\eqref{eqn:concave_init_decomposition}, we have, with probability at least $1-2\delta$:
\begin{align*}
\| f_0 - f^* - \hat{g}_k \|_P^2 - \|f_0 - f^* - g^*_k \|_P^2 \leq
   c B^2 \tilde{\sigma}^{1/2} 
     \sqrt[4]{ \frac{s^5}{n^{4/5}} \log^2 \frac{4np}{\delta}} 
\end{align*}
Taking a union bound across the $s$ dimensions completes the result.
\end{proof}








\subsubsection{Support Lemmas}

%%%%%%%%%%
%% Uniform convergence lemma
%%
%%
%%
%%
%%%%%%%%%%%%%%%%%%%%

\begin{lemma}
\label{lem:uniform_convergence}
With probability at least $1-\delta$:
\begin{align*}
\sup_{f \in \mathcal{C}^s_L} \Big| \| f_0 - f \|^2_n - \|f_0 - f \|^2_P\Big| \leq
   c B^3 \sqrt{ \frac{s^5b^{1/2}}{n^{4/5}} \log \frac{2}{\delta}}
\end{align*}
\end{lemma}

\begin{proof}
Let $\mathcal{G}$ denote the off-centered set of convex functions, that is, $\mathcal{G} \equiv \mathcal{C}^s - f_0$. Note that if $h \in \mathcal{G}$, then $\| h \|_\infty = \| f_0 - f \|_\infty \leq 4 s B$.
There exists an $\epsilon$-bracketing of $\mathcal{G}$, and
%For every $h \in \mathcal{G}$, there exists $h^L$ in the bracketing such that $\| h^U - h^L \|_{L_1(P)} \leq \epsilon$. 
by Corollary~\ref{cor:convexadditive_lp}, the bracketing has size at most $\log N_{[]}(2\epsilon, \mathcal{C}^s, L_1(P)) \leq s K^{**}\left( \frac{2sbB}{\epsilon} \right)^{1/2}$. By Corollary~\ref{cor:convexbracket_ln}, we know that with probability at least $1-\delta$, we have that $\|h_U - h_L\|_{L_1(P_n)} \leq \epsilon + \epsilon_{n,\delta}$ for all pairs $(h_U, h_L)$ in the bracketing, where $\epsilon_{n,\delta} = sB \sqrt{ \frac{K^{**} (2sbB)^{1/2} \log \frac{2}{\delta}}{2 \epsilon^{1/2} n}}$.
For a particular function $h \in \mathcal{G}$, we can construct $\psi_L \equiv \min( |h_U|, |h_L|)$ and $\psi_U \equiv \max( |h_U|, |h_L| )$ so that
\[
\psi_L^2 \leq h^2 \leq \psi_U^2.
\]
We can then bound the $L_1(P)$ norm of $\psi_U^2 - \psi_L^2$ as
\begin{align*}
\int (\psi_U^2(x) - \psi_L^2(x)) p(x)dx  &\leq  \int | h_U^2(x) - h_L^2(x)| p(x) dx \\
   &\leq \int | h_U(x) - h_L(x) | \, |h_U(x) + h_L(x)| p(x) dx \\
   &\leq 2sB \epsilon
\end{align*}

% And, with probability at least $1-\delta$, for all $\psi_U, \psi_L$:
% \begin{align*}
% \frac{1}{n} \sum_{i=1}^n ( \psi_U^2(X_i) - \psi_L^2(X_i)) &\leq 
%          \frac{1}{n} \sum_{i=1}^n | h_U^2(X_i) - h_L^2(X_i)| \\
%   &\leq \frac{1}{n}\sum_{i=1}^n | h_U(X_i) - h_L(X_i)| \, |h_U(X_i) + h_L(X_i)| \\
%   &\leq 2sB(\epsilon + \epsilon_{n,\delta})
% \end{align*}
Now we can bound $\| h \|_n^2 - \| h \|_P^2$ as
\begin{align}
\frac{1}{n} \sum_{i=1}^n \psi_L(X_i)^2 - \E \psi_U(X)^2  \leq
    \| h \|_n^2 - \| h \|_P^2 \leq
  \frac{1}{n} \sum_{i=1}^n \psi_U(X_i)^2 - \E \psi_L(X)^2  \label{eqn:hpsi_bound}
\end{align}
Since 
$\psi_L(X_i)^2$ and $\psi_U(X_i)^2$ are bounded random variables with
upper bound $(sB)^2$, Hoeffding's inequality and union bound give that,
with probability at least $1-\delta$,, for all $\psi_L$ (and likewise $\psi_U$)
\[
\left| \frac{1}{n} \sum_{i=1}^n \psi_L(X_i)^2 - 
   \E \psi_L(X)^2 \right| \leq (sB)^2 \sqrt{ \frac{ sK^{**} (sBb)^{1/2} \log \frac{2}{\delta}}{ \epsilon^{1/2} n} }
\]

Plugging this into equation~\eqref{eqn:hpsi_bound} above, we have that:
\begin{align*}
& \E \psi_L(X)^2 - \E \psi_U(X)^2 - 
(sB)^2 \sqrt{ \frac{ sK^{**} (sBb)^{1/2} \log \frac{2}{\delta}}{ \epsilon^{1/2} n} } \\
 & \leq 
 \| h \|_n^2 - \| h \|_P^2 \leq
\E \psi_U(X)^2 - \E \psi_L(X)^2 + 
(sB)^2 \sqrt{ \frac{ sK^{**} (sBb)^{1/2} \log \frac{2}{\delta}}{ \epsilon^{1/2} n} }.
\end{align*}
Using the $L_1(P)$ norm of $\psi_U^2 - \psi_L^2$ result, we have
\begin{align*}
-sB\epsilon - 
(sB)^2 \sqrt{ \frac{ sK^{**} (sBb)^{1/2} \log \frac{2}{\delta}}{ \epsilon^{1/2} n} } \leq 
 \| h \|_n^2 - \| h \|_P^2 \leq
sB\epsilon + 
(sB)^2 \sqrt{ \frac{ sK^{**} (sBb)^{1/2} \log \frac{2}{\delta}}{ \epsilon^{1/2} n} }
\end{align*}

We balance the terms by choosing $\epsilon = \left( \frac{ (sB)^2 sK^{**} (sBb)^{1/2}}{n} \right)^{2/5}$.
We have then that, with probability at least $1-\delta$,
\begin{align*}
\big| \| h \|_n^2 - \| h \|_P^2  \big| \leq
  c B^3 \sqrt{ \frac{s^5 b^{1/2} \log \frac{2}{\delta}}{n^{4/5}}}
\end{align*}

%For a $h \in \mathcal{G}$, let $h_\epsilon$ denote a function in the $\epsilon$-bracketing of $\mathcal{G}$ closest to $h$. It obviously must be that $\| h - h_\epsilon \|_n \leq \| h - h_\epsilon \|_\infty \leq \epsilon$.

% Because $\| h \|_n = \| h - h^L + h^L \|_n$, we have that
% \begin{align*}
% \|h^L \|_n - \| h - h^L \|_n &\leq 
%     \| h \|_n \leq \|h^L \|_n + \| h - h^L \|_n \\
% \|h^L \|_n - \epsilon - \epsilon_{n,\delta} &\leq 
%     \| h \|_n \leq \|h^L \|_n + \epsilon + \epsilon_{n,\delta}  \\
% \| h^L \|^2_n - 8 \epsilon (sB)  &\leq \| h  \|^2_n  \leq
%    \| h^L \|_n^2 + 8 \epsilon (sB) 
% \end{align*}
% where we used the fact that $\| h - h^L \|_n \leq \| h - h^L \|_\infty \leq \epsilon$ and $ \| h^L \|_n \leq \| h^L \|_\infty \leq 4 s L b$.


% And likewise:
% \begin{align*}
% \| h_\epsilon \|^2_P - 8 \epsilon (s L b)  &\leq \| h  \|^2_P \leq
%    \| h_\epsilon \|_P^2 + 8 \epsilon (s L b)
% \end{align*} 

% Therefore, 
% \begin{align*}
% \sup_{h \in \mathcal{G}}  \Big| \|h\|^2_n - \|h\|^2_P \Big| \leq 
% \sup_{h_\epsilon} \Big| \|h_\epsilon\|^2_n - \|h_\epsilon\|^2_P \Big| 
%         + \epsilon (16 sLb)
% \end{align*}

% Since $\| h_\epsilon \|_n^2 = \frac{1}{n} \sum_{i=1}^n h_\epsilon(X_i)^2$ is an average of bounded random variables, we have by Union Bound and Hoeffding Inequality that, with probability at most $1 - \delta$,
% \begin{align*}
% \sup_{h_\epsilon} \Big| \|h_\epsilon\|^2_n - \|h_\epsilon\|^2_P \Big| &\leq
%   (8sLb)^2 \sqrt{ \frac{1}{cn} \big(\log \frac{2}{\delta} 
%         + \log N_\infty(\epsilon, \mathcal{C}_L^s) \big) } \\
% &\leq (8sLb)^2 \sqrt{ \frac{1}{cn} \big(\log \frac{2}{\delta} 
%         + C s^{1.5} Lb \epsilon^{-1/2}   \big) }
% \end{align*}

% We will set $\epsilon = \frac{1}{n^{2/5}} (C s^{0.5} L b)^2$. 
% Therefore:
% \begin{align*}
% \sup_{h_\epsilon} \Big|  \|h_\epsilon\|^2_n - \|h_\epsilon\|^2_P \Big| &\leq
%    (8sLb)^2 \sqrt{ \frac{1}{cn} \big(\log \frac{2}{\delta} 
%         + s n^{1/5}  \big) } \\
%   &\leq (8Lb)^2 \sqrt{ \frac{s^5}{cn^{4/5}} \log \frac{2}{\delta} }
% \end{align*}
% And
% \begin{align*}
% \sup_{h \in \mathcal{G}}  \Big| \|h\|^2_n - \|h\|^2_P \Big| &\leq 
% \sup_{h_\epsilon} \Big| \|h_\epsilon\|^2_n - \|h_\epsilon\|^2_P \Big| 
%         + \frac{1}{n^{2/5}} C^2 s^2 (Lb)^3 \\
%   &\leq (8Lb)^2 \sqrt{ \frac{s^5}{c n^{4/5}} \log \frac{2}{\delta}}
%      + (CLb)^2 \sqrt{ \frac{s^4}{n^{4/5}} } \\
%   &\leq c (Lb)^3 \sqrt{ \frac{s^5}{n^{4/5}} \log \frac{2}{\delta}}
% \end{align*}

\end{proof}



\begin{lemma}
\label{lem:remove_centering}
Let $f_0$ and $f^*$ be defined as in Section~\ref{sec:false_negative_proof_notations}. Define $\bar{f}^* = \frac{1}{n} \sum_{i=1}^n f^*(X_i)$.
Then, with probability at least $1 - 2\delta$,
\[
\Big | \| f_0 - f^* \|_n^2 - \| f_0 - f^* + \bar{f}^* \|_n^2 \Big| \leq
    c (sB)^2 \frac{1}{n} \log \frac{4}{\delta}
\]
\end{lemma}

\begin{proof}
We decompose the empirical norm as
\begin{align*}
\| f_0 - f^* + \bar{f}^* \|_n^2 &= \| f_0 - f^* \|_n^2 
    + 2 \langle f_0 - f^*, \bar{f}^* \rangle + \bar{f}^{*2} \\
  &= \| f_0 - f^* \|_n^2 + 2 \bar{f}^* \langle f_0 - f^*, \mathbf{1} \rangle_n + 
    \bar{f}^{*2} \\
  &= \| f_0 - f^* \|_n^2 + 2 \bar{f}^* \bar{f}_0 - \bar{f}^{*2}.
\end{align*}
Now
$\bar{f}^* = \frac{1}{n} \sum_{i=1}^n f^*(X_i)$ is the average of $n$ bounded mean-zero random variables and therefore, with probability at least $1-\delta$, $| \bar{f}^* | \leq 4 sB \sqrt{ \frac{1}{n} \log \frac{2}{\delta} }$.
The same reasoning likewise applies to $\bar{f}_0 = \frac{1}{n} \sum_{i=1}^n f_0(X_i)$.

Taking a union bound and we have that, with probability at least $1- 2\delta$, 
\begin{align*}
| \bar{f}^* | | \bar{f}_0 | &\leq c (sB)^2 \frac{1}{n} \log \frac{2}{\delta} \\
\bar{f}^{*2} &\leq c (sB)^2 \frac{1}{n} \log \frac{2}{\delta}
\end{align*}
Therefore, with probability at least $1 - 2\delta$,
\[
\|f_0 - f^*\|_n^2 - c (sB)^2 \frac{1}{n} \log \frac{2}{\delta} \leq
    \| f_0 - f^* + \bar{f}^* \|_n^2 \leq 
\|f_0 - f^*\|_n^2 + c (sB)^2 \frac{1}{n} \log \frac{2}{\delta}
\]
\end{proof}




%%%%%%%%%%%%%
%% Technical Material
%%
%%
%%
%%
%%%%%%%%%%%%%%%%%%%% 

% \subsection{Supporting Technical Material}
 
\subsubsection{Supporting Lemma for Theorem~\ref{thm:acdc_faithful}}

\begin{lemma}
\label{lem:acdc_derivative_bound}
Let $f : [0,1]^p \rightarrow \R$ be a twice differentiable function. Suppose $p(\mathbf{x})$ is a density on $[0,1]^p$ such that $\partial_{x_k} p(\mathbf{x}_{-k} \given x_k)$ and $\partial_{x_k}^2 p(\mathbf{x}_{-k} \given x_k)$ are all continuous as functions of $x_k$. Let $\phi(\mathbf{x}_{-k})$ be a continuous function not dependent on $x_k$.

Then, $h^*_k(x_k) \equiv \E[ f(X) - \phi(X_{-k}) \given x_k]$ is twice
differentiable and has a second derivative lower 
bounded away from $-\infty$.
\end{lemma}

\begin{proof}
We can write
\[
h^*_k(x_k) = \int_{\mathbf{x}_{-k}} \big(f(\mathbf{x}) - \phi(\mathbf{x}_{-k})\big) p(\mathbf{x}_{-k} \given x_k) d \mathbf{x}_{-k}
\]
The integrand is bounded because it is a sum-product of continuous functions over a compact set. Therefore, we can differentiate under the integral and derive that
\begin{align*}
\partial_{x_k} h^*_k(x_k) &= 
    \int_{\mathbf{x}_{-k}} f'(\mathbf{x}) p(\mathbf{x}_{-k} \given x_k) + (f(\mathbf{x}) - \phi(\mathbf{x}_{-k}) p'(\mathbf{x}_{-k} \given x_k) d \mathbf{x}_{-k}\\
\partial_{x_k}^2 h^*_k(x_k) &= 
    \int_{\mathbf{x}_{-k}} f''(\mathbf{x}) p(\mathbf{x}_{-k} \given x_k) + 2f'(\mathbf{x}) p'(\mathbf{x}_{-k} \given x_k)  + (f(\mathbf{x}) - \phi(\mathbf{x}_{-k}) p''(\mathbf{x}_{-k} \given x_k) d \mathbf{x}_{-k}
\end{align*}
where we have used the shorthand $f'(\mathbf{x}), p'(\mathbf{x}_{-k}
\given x_k)$ to denote $\partial_{x_k} f(\mathbf{x})$, $\partial_{x_k}
p(\mathbf{x}_{-k} \given x_k)$, etc.

This proves that $h^*_k(x_k)$ is twice-differentiable. To see that the second derivative is lower bounded, we note that $f''(\mathbf{x}) p(\mathbf{x}_{-k} \given x_k)$ is non-negative and the other terms in the second-derivative are all continuous functions on a compact set and thus bounded. 
\end{proof}


subsubsection{Concentration of Measure}

A sub-exponential random is the square of a sub-Gaussian random
variable \cite{vershynin2010introduction}.

\begin{proposition} (Subexponential Concentration
  \cite{vershynin2010introduction})
Let $X_1,...,X_n$ be zero-mean independent subexponential random
variables with subexponential scale $K$. 
Then
\[
P( | \frac{1}{n} \sum_{i=1}^n X_i | \geq \epsilon) \leq
	2 \exp \left[ -c n \min\left( \frac{\epsilon^2}{K^2}, \frac{\epsilon}{K} \right) \right]
\]
where $c > 0$ is an absolute constant.
\end{proposition}

For uncentered subexponential random variables, we can use the following fact. If $X_i$ subexponential with scale $K$, then $X_i - \E[X_i]$ is also subexponential with scale at most $2K$.
Restating, we can set
\[
c \min\left( \frac{\epsilon^2}{K^2}, \frac{\epsilon}{K} \right) = \frac{1}{n} \log \frac{1}{\delta}.
\]
Thus, with probability at least $1-\delta$, the deviation is at most
\[
K \max\left( \sqrt{\frac{1}{cn} \log \frac{C}{\delta}},  \frac{1}{cn} \log \frac{C}{\delta} \right).
\]


\begin{corollary}
Let $W_1,...,W_n$ be $n$ independent sub-Gaussian random variables with sub-Gaussian scale $\sigma$. 
Then, for all $n > n_0$, with probability at least $1- \frac{1}{n}$,
\[
\frac{1}{n} \sum_{i=1}^n W_i^2 \leq c \sigma^2 .
\]
\end{corollary}

\begin{proof}
Using the subexponential concentration inequality, we know that, with probability at least $1-\frac{1}{n}$, 

\[
\left| \frac{1}{n} \sum_{i=1}^n W_i^2 - \E W^2\right| \leq \sigma^2 \max\left( \sqrt{\frac{1}{cn} \log \frac{C}{\delta}}, \frac{1}{cn}\log \frac{C}{\delta} \right).
\]

First, let $\delta = \frac{1}{n}$. Suppose $n$ is large enough such that $ \frac{1}{cn} \log Cn < 1$. Then, we have, with probability at least $1-\frac{1}{n}$,
\begin{align*}
 \frac{1}{n} \sum_{i=1}^n W_i^2 &\leq c\sigma^2 \Big(1+\sqrt{\frac{1}{cn} \log Cn}\Big) \\
		&\leq 2 c \sigma^2.
 \end{align*}
\end{proof}



\subsubsection{Sampling Without Replacement}

\begin{lemma} (\cite{serfling1974probability}) 
Let $x_1,..., x_N$ be a finite list, $\bar{x} = \mu$. Let $X_1,...,X_n$ be sampled from $x$ without replacement. 

Let $b = \max_i x_i$ and $a = \min_i x_i$. Let $r_n = 1- \frac{n-1}{N}$. Let $S_n = \sum_i X_i$.
Then we have that
\[
\P( S_n - n \mu \geq n \epsilon) \leq \exp\left( - 2 n \epsilon^2 \frac{1}{r_n (b-a)^2}\right).
\]
\end{lemma}

\begin{corollary}
\label{cor:serfling}
Suppose $\mu = 0$. 
\[
\P\left( \frac{1}{N} S_n \geq \epsilon\right) \leq \exp\left( -2 N \epsilon^2 \frac{1}{(b-a)^2}\right)
\]
And, by union bound, we have that
\[
\P\left( | \frac{1}{N} S_n| \geq \epsilon\right) \leq 2 \exp\left( -2 N \epsilon^2 \frac{1}{(b-a)^2}\right)
\]

\end{corollary}

A simple restatement is that with probability at least $1- \delta$, the deviation $| \frac{1}{N} S_n|$ is at most $ (b-a) \sqrt{ \frac{1}{2N} \log \frac{2}{\delta}}$.

\begin{proof}
\[
\P\left( \frac{1}{N} S_n \geq \epsilon\right) = \P\left( S_n \geq \frac{N}{n} n \epsilon\right) \leq \exp\left( - 2 n \frac{N^2}{n^2} \epsilon^2 \frac{1}{r_n (b-a)^2} \right).
\]
We note that $r_n \leq 1$ always, and $n \leq N$ always.   Thus,
\[
\exp\left( - 2 n \frac{N^2}{n^2} \epsilon^2 \frac{1}{r_n (b-a)^2} \right)  \leq \exp\left( - 2 N \epsilon^2 \frac{1}{(b-a)^2}\right)
\]
completing the proof.
\end{proof}

\subsubsection{Bracketing Numbers for Convex Functions}

\begin{definition}
% upper, lower, epsilon room
% metric
% function space
Let $\mathcal{C}$ be a set of functions. For a given $\epsilon$ and metric $\rho$ (which we take to be $L_2$ or $L_2(P)$), we define a \textit{bracketing} of $\mathcal{C}$ to be a set of pairs of functions $\{ (f_L, f_U) \}$ satisfying (1) $\rho( f_L, f_U) \leq \epsilon$ and (2) for any $f \in \mathcal{C}$, there exist a pair $(f_L, f_U)$ where $f^U \geq f \geq f^L$. 

We let $N_{[]}(\epsilon, \mathbf{C}, \rho)$ denote the size of the smallest bracketing of $\mathcal{C}$
\end{definition}

\begin{proposition} (Proposition 16 in \cite{kim2014global})
\label{prop:convexbracket}
Let $\mathcal{C}$ be the set of convex functions supported on $[-b, b]^d$ and uniformly bounded by $B$. Then there exist constants $\epsilon_3$ and $K^{**}$, dependent on $d$, such that
\[
\log N_{[]} (2\epsilon, \mathcal{C}, L_2) \leq K^{**} \left( \frac{2bB}{\epsilon} \right)^{d/2}
\]
for all $\epsilon \in (0, B \epsilon_3]$.
\end{proposition}

It is trivial to extend Kim and Samworth's result to the $L_2(P)$ norm for an absolutely continuous distribution $P$.

\begin{proposition}
\label{prop:convexbracket_lp}
Let $P$ be a distribution with a density $p$. Let $\mathcal{C}, b, B, \epsilon_3, K^{**}$ be defined as in Proposition~\ref{prop:convexbracket}. Then,
\[
\log N_{[]} (2\epsilon, \mathcal{C}, L_1(P)) \leq K^{**} \left( \frac{2bB}{\epsilon} \right)^{d/2}
\]
for all $\epsilon \in (0, B\epsilon_3]$.
\end{proposition}

\begin{proof}
Let $\mathcal{C}_\epsilon$ be the bracketing the satisfies the size bound in Proposition~\ref{prop:convexbracket_lp}. 

Let $(f_L, f_U) \in \mathcal{C}_\epsilon$. Then we have that:
\begin{align*}
\| f_L - f_U \|_{L_1(P)} &= \int | f_L(x) - f_U(x)| p(x) dx \\
   &\leq \left( \int | f_L(x) - f_U(x) |^2 dx \right)^{1/2}
      \left( \int p(x)^2 dx \right)^{1/2} \\
  &\leq \left( \int | f_L(x) - f_U(x)|^2 dx \right)^{1/2}\\
 &\leq \| f_L - f_U \|_{L_2} \leq \epsilon.
\end{align*}
On the third line, we used the fact that $\int p(x)^2 dx \leq \left( \int p(x) dx \right)^2 \leq 1$.
\end{proof}

It is also simple to extend the bracketing number result to additive convex functions. As before, let $\mathcal{C}^s$ be the set of additive convex functions with $s$ components.

\begin{corollary}
\label{cor:convexadditive_lp}
Let $P$ be a distribution with a density $p$ upper bounded by $c_u$. Let $b, B, \epsilon_3, K^{**}$ be defined as in Proposition~\ref{prop:convexbracket}. Then,
\[
\log N_{[]}(2\epsilon, \mathcal{C}^s, L_1(P)) \leq s K^{**} 
    \left( \frac{2sbB}{\epsilon} \right)^{1/2}
\]
for all $\epsilon \in (0, s B \epsilon_3]$.
\end{corollary}

\begin{proof}
Let $f \in \mathcal{C}^s$. We can construct an $\epsilon$-bracketing for $f$ through $\epsilon/s$-bracketings for each of the components $\{ f_k \}_{k=1,...,s}$:
\[f_U = \sum_{k=1}^s f_{Uk}  \qquad f_L = \sum_{k=1}^s f_{Lk} \]
It is clear that $f_U \geq f \geq f_L$. It is also clear that $\| f_U - f_L \|_{L_1(P)} \leq \sum_{k=1}^s \| f_{Uk} - f_{Lk} \|_{L_1(P)} \leq \epsilon$.
\end{proof}


The following result follows from Corollary~\ref{prop:convexbracket_lp} directly by a union bound. 

\begin{corollary}
\label{cor:convexbracket_ln}
Let $X_1,...,X_n$ be random samples from a distribution $P$. Let $1 > \delta > 0$. Let $\mathcal{C}^s_\epsilon$ be an $\epsilon$-bracketing of $\mathcal{C}^s$ with respect to the $L_1(P)$-norm whose size is at most $N_{[]}( 2\epsilon, \mathcal{C}^s, L_1(P))$. Let $\epsilon \in (0, s B \epsilon_3]$.

Then, with probability at least $1-\delta$, for all pairs $(f_L, f_U) \in \mathcal{C}^s_\epsilon$, we have that
\begin{align*}
\frac{1}{n} \sum_{i=1}^n |f_L(X_i) - f_U(X_i)| &\leq \epsilon + \epsilon_{n, \delta}
\end{align*}
where 
$$\epsilon_{n,\delta} \equiv 
sB \sqrt{ \frac{ \log N_{[]}(2\epsilon, \mathcal{C}^s, L_2(P)) + \log \frac{1}{\delta}}{2n}} 
= \sqrt{ \frac{ sK^{**}(sBb)^{1/2}}{2\epsilon^{1/2}n} + \frac{1}{2n} \log \frac{1}{\delta}}.$$
\end{corollary}

\begin{proof}
Noting that $|f_L(X_i) - f_U(X_i)|$ is at most $sB$ and there are
$N_{[]}(2\epsilon, \mathcal{C}^s, L_1(P))$ pairs $(f_L, f_U)$, the
inequality follows from a direct application of a union bound and Hoeffding's Inequality.
\end{proof}

To make the expression in this corollary easier to work with, we derive an upper bound for $\epsilon_{n, \delta}$. Suppose 
\begin{align}
\epsilon^{1/2} \leq 2sK^{**} (sBb)^{1/2} \quad \trm{and} \quad \log \frac{1}{\delta} \geq 2 \label{cond:simplify_covering_number}.
\end{align}
Then we have that
\begin{align*}
\epsilon_n \leq sB \sqrt{ \frac{ sK^{**} (sBb)^{1/2} \log \frac{1}{\delta}}{\epsilon^{1/2}n}}.
\end{align*}

\section{Gaussian Example}
\label{sec:gaussian_example}

Let $H$ be a positive definite matrix and let $f(x_1, x_2) = H_{11} x_1^2 + 2H_{12} x_1x_2 + H_{22} x_2^2 + c$ be a quadratic form where $c$ is a constant such that $\E[f(X)] = 0$. Let $X \sim N(0, \Sigma)$ be a random bivariate Gaussian vector with covariance $\Sigma = [1, \alpha; \alpha, 1]$ 

\begin{proposition}
Let $f^*_1(x_1) + f^*_2(x_2)$ be the additive projection of $f$ under the bivariate Gaussian distribution. That is,
\begin{align*} 
f^*_1, f^*_2 \equiv \argmin_{f_1, f_2} \Big\{ \E \left(f(X) - f_1(X_1) - f_2(X_2)\right)^2 \,:\, \E[f_1(X_1)] = \E[f_2(X_2)] = 0 \Big\}
\end{align*}

Then, we have that 
\begin{align*}
f^*_1(x_1) &= \left( \frac{T_1 - T_2 \alpha^2}{1 - \alpha^4} \right) x_1^2 + c_1 \\
f^*_2(x_2) &= \left( \frac{T_2 - T_1\alpha^2}{1 - \alpha^4} \right) x_2^2 + c_2
\end{align*}
where $T_1 = H_{11} + 2H_{12} \alpha + H_{22} \alpha^2$ and $T_2 = H_{22} + 2H_{12} \alpha + H_{11} \alpha^2$ and $c_1,c_2$ are constants such that $\E[f_1^*(X_1)] = \E[f_2^*(X_2)] = 0$.
\end{proposition}

\begin{proof}
By Lemma~\ref{lem:general_int_reduction}, we need only verify that $f^*_1, f^*_2$ satisfy
\begin{align*}
f^*_1(x_1) &= \E[ f(X) - f_2^*(X_2) \given x_1] \\
f^*_2(x_2) &= \E[ f(X) - f_1^*(X_1) \given x_2 ] .
\end{align*}
Let us guess that $f^*_1, f^*_2$ are quadratic forms $f^*_1(x_1) = a_1 x_1^2 + c_1$, $f^*_2(x_2) = a_2 x_2^2 + c_2$ and verify that there exist $a_1, a_2$ to satisfy the above equations. Since we are not interested in constants, we define $\simeq$ to be equality up to a constant. 
Then, 
\begin{align*}
&\E[ f(X) - f_2^*(X_2) \given x_1]\\
& \simeq \E[ H_{11} X_1^2 + 2H_{12}X_1 X_2 + H_{22}X_2^2 - a_2 X_2^2 \given x_1 ] \\ 
    &\simeq H_{11} x_1^2 + 2 H_{12} x_1 \E[X_2 \given x_1] + H_{22} \E[X_2^2 \given x_1] - a_2\E[X_2^2 \given x_1] \\
   &\simeq H_{11} x_1^2 + 2H_{12} \alpha x_1^2 + H_{22} \alpha^2 x_1^2 - a_2 \alpha^2 x_1^2\\
   &\simeq (H_{11} + 2 H_{12} \alpha + H_{22} \alpha^2 - a_2 \alpha^2) x_1^2.
\end{align*}
Likewise, we have that
\[
\E[ f(X) - f_1^2(X_1) \given x_2] \simeq (H_{22} + 2H_{12}\alpha + H_{22}\alpha^2 - a_1 \alpha^2) x_2^2.
\]
Thus, $a_1, a_2$ need only satisfy the linear system
\begin{align*}
T_1 - a_2 \alpha^2 &= a_1 \\
T_2 - a_1 \alpha^2 &= a_2 
\end{align*}
where $T_1 = H_{11} + 2H_{12} \alpha + H_{22} \alpha^2$ and $T_2 = H_{22} + 2H_{12} \alpha + H_{11} \alpha^2$.
It is then simple to solve the system and verify that $a_1, a_2$ are as specified.
\end{proof}


% \subsubsection{Covering Number for Lipschitz Convex Functions}

% \begin{definition}
% $\{ f_1,..., f_N\} \subset \mathcal{C}[b,B,L]$ is an $\epsilon$-covering of $\mathcal{C}[b,B,L]$ if for all $f \in \mathcal{C}[b,B,L]$, there exist $f_i$ such that $\| f - f_i \|_\infty \leq \epsilon$.

% We define $N_\infty( \epsilon, \mathcal{C}[b,B,L])$ as the size of the minimum covering.
% \end{definition}

% \begin{lemma} (Bronshtein 1974)
% \[
% \log N_\infty (\epsilon, \mathcal{C}[b,B,L]) \leq C\left( \frac{bBL}{\epsilon} \right)^{1/2}
% \]
% For some absolute constant $C$.
% \end{lemma}

% \begin{lemma}
% \[
% \log N_\infty( \epsilon, \mathcal{C}^s[b,B,L])  \leq C s \left(\frac{bBLs}{\epsilon}\right)^{1/2}
% \]
% For some absolute constant $C$.
% \end{lemma}

% \begin{proof}
% Let $f = \sum_{k=1}^s f_k$ be a convex additive function. Let $\{ f'_k \}_{k=1,..,s}$ be $k$ functions from a $\frac{\epsilon}{s}$ $L_\infty$ covering of $\mathcal{C}[b,B,L]$. 

% Let $f' \coloneqq \sum_{k=1}^s f'_k$, then 
% \[
% \| f' - f \|_{\infty} \leq \sum_{k=1}^s \| f_k - f'_k \|_\infty \leq s \frac{\epsilon}{s} \leq \epsilon
% \]

% Therefore, a product of $s$ $\frac{\epsilon}{s}$-coverings of univariate functions induces an $\epsilon$-covering of the additive functions.
% \end{proof}

% In our proofs, we will use the Dudley's chaining: [TODO:cite van de geer]

% \begin{theorem} (Dudley's Chaining) \\
% \label{thm:chaining}
% Let $\mathcal{G} = \{ g : \| g \|_n \leq R \}$. Let $M( \epsilon, R)$ be the size of the minimal $\epsilon$-covering of $\mathcal{G}$ with respect to the $\| \cdot \|_n$ norm. Suppose $w = (w_1, ..., w_n)$ is a vector of i.i.d. sub-Gaussian random variables with scale $\sigma$.\\

% Suppose $\delta > 0$ is such that
% \[
% \sqrt{n} \delta \geq \sigma \Big( 
%    14 \sum_{s=0}^\infty 2^{-s} \sqrt{ \log M( 2^{-s} R, \mathcal{G})} 
%       \Big)\vee 70 \log 2 R
% \]
% Then we have that 
% \[
% P\Big( \sup_{g \in \mathcal{G}} \langle w, g \rangle_n \geq \delta \Big) \leq
%   4 \exp \Big( - \frac{n \delta^2}{(70R)^2\sigma^2} \Big)
% \]
% \end{theorem}

% For convenience, we can upper bound the metric-entropy sum with an integral: \\
% $\ds \sum_{s=0}^\infty 2^{-s} \sqrt{ \log M( 2^{-s} R, \mathcal{G})} 
%   \leq \int_0^R \sqrt{\log M(t, \mathcal{G}) } dt
% $




% DO NOT CHANGE; RefTex variables -minx
 
%%% Local Variables: ***
%%% mode:latex ***
%%% TeX-master: "paper.tex" ***
%%% End: ***

