
\documentclass[pdftex,12pt]{article}

\usepackage{hyperref}
\usepackage{fullpage}
\usepackage{enumerate}
\usepackage{amsfonts,amsmath,amssymb,amsthm}
\usepackage{graphicx,subfigure}
\usepackage[pdftex]{color}
\usepackage{epic,eepic,eepicemu}
\usepackage{epsf}
\usepackage{epsfig}
\usepackage{fancyhdr}
\usepackage{graphics}
\usepackage{psfrag,latexsym}
\usepackage{times}
\newcommand{\thetamin}{\ensuremath{\theta^*_{min}}}
\newcommand{\mutinc}{\ensuremath{\alpha}}
\bibliographystyle{abbrv}


\setlength{\textwidth}{\paperwidth}
\addtolength{\textwidth}{-6cm}
\setlength{\textheight}{\paperheight}
\addtolength{\textheight}{-4cm}
\addtolength{\textheight}{-1.1\headheight}
\addtolength{\textheight}{-\headsep}
\addtolength{\textheight}{-\footskip}
\setlength{\oddsidemargin}{0.5cm}
\setlength{\evensidemargin}{0.5cm}
\def\P{{\mathbb P}}
\def\E{{\mathbb E}}
\def\degmax{d}
\def\pdim{p}
\def\numobs{n}
\let\hat\widehat
\definecolor{blue}{rgb}{0.01,0.01,0.75}
\def\rc#1{{\it\textcolor{blue}{#1}}\smallskip}
\parindent0pt
\parskip12pt

%%%%%%%%%%%%%%%%%%%%%%%%%%%%%%%%%%%%%%%%%%%%%%%%%%%%%%%%%%%%%%%%%%%%%%%%%%
\renewcommand{\baselinestretch}{1.02}

\begin{document}

\vspace*{5pt}

As before, we are grateful to the reviewers for their careful reading
and very helpful editing and suggestions. We have fixed all of the typos
found by the reviewers. We also have improved the paper in other minor
aspects in response to the reviews---these changes are described
below.

\subsection*{Review 4}


\begin{enumerate}
\item \rc{On page 30, I did not understand : ”we set $K = 7$ for all the experiments”, and then ”We set $s = 5$” and ”the true variables are $X_j$ for $j = 5,6,7,8,9,10$”.}

Here, the true function is a soft-max of linear forms $\beta_k^T x$, i.e., 
$$f_0(x) = \log \left( \sum_{k=1}^K \exp( \beta_k^T x) \right) - \mu$$
where the $\beta_k$s are randomly generated unit vectors. $K$ is the
number of linear pieces and $s$ is the number of relevant
variables. We have clarified this a bit more in Section 6 of the
paper.
\end{enumerate}


\subsection*{Reviewer 5}

\begin{enumerate}
\item \rc{In Section 2.3, the authors preview their main results concerning variable selection consistency in a finite-sample setting. However, at this point, the “true regression function” $f_0$ has not yet been defined (it is defined later in Section 5.2). It would be helpful if the authors were to state the model generating the data in this section.}

We have stated the model in the beginning of Section 2.3.

\item \rc{In Example 3.3, the authors describe a method for approximating any bounded density over a hypercube arbitrarily well using boundary flat densities. The second paragraph mentions truncating the density to the hypercube $[\epsilon, 1−\epsilon]^p$—since the truncated function isn’t a density, it should technically also be rescaled.}

This has been fixed. 

\item \rc{In Assumption A4 at the bottom of page 25, the moment generating function should read $\E e^{tW}$ rather than $\E e^{t \epsilon}$. There is some change in notation in Section 6.1, where $W$ is replaced by $\epsilon$; perhaps it would be better to keep the notation consistent.}

We have changed $\epsilon$ to $W$ across the paper to keep the notation consistent. 

\item \rc{The last sentence of the first paragraph on page 26 is a bit hard to follow. What does “if $X$ is independent” mean (does this mean $p(x)$ is a product distribution)? I did not understand where the simplification came from.}

``$X$ is independent'' does mean that $p(x)$ is a product distribution; we have clarified this in the paper. We have also added a new section to the supplement (Section 11, Proposition 11.1) that explains and derives the simplification. 

\item \rc{The $\nu = 0.9$ curve in Figure 5(d) is rather ill-behaved compared to the other curves. However, if my understanding is correct, the probability of successful screening should still tend to 1 in this case. Perhaps the authors could consider plotting more samples to show eventual consistency for this curve, as well (and also take an average over more trials, since the behavior of the curve is somewhat noisy).}

The probability of successful screening does tend to one. That was not
reflected in the old plot because the old plot describes the
probability both that the number of false positives is zero (successful
screening) and that the number of false positives is less than 20. It
was because of this latter criterion that the $\nu =0.9$ curve did not
tend to one in the old plot. The threshold of 20 was arbitrarily
chosen; we have removed this restriction and revised the plots. The updated figures
5(d) and 5(e) show only the probability that the number of false
negatives is zero; the number of false positives is reflected in the
newly added figures 6(f) and 6(g), which show that total number of
variables selected in the experiments that correspond to figures 5(d)
and 5(e). We have also updated the discussion in Sections 6.1.3 and
6.1.4.

\end{enumerate}




\vspace*{10pt}

Sincerely, 


Min Xu, Minhua Chen, and John Lafferty\\[1pt]
\today{}

\bibliography{local}
\end{document}
