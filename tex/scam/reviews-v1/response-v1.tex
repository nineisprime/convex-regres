
\documentclass[pdftex,12pt]{article}

\usepackage{hyperref}
\usepackage{fullpage}
\usepackage{enumerate}
\usepackage{amsfonts,amsmath,amssymb,amsthm}
\usepackage{graphicx,subfigure}
\usepackage[pdftex]{color}
\usepackage{epic,eepic,eepicemu}
\usepackage{epsf}
\usepackage{epsfig}
\usepackage{fancyhdr}
\usepackage{graphics}
\usepackage{psfrag,latexsym}
\usepackage{times}
\newcommand{\thetamin}{\ensuremath{\theta^*_{min}}}
\newcommand{\mutinc}{\ensuremath{\alpha}}
\bibliographystyle{abbrv}


\setlength{\textwidth}{\paperwidth}
\addtolength{\textwidth}{-6cm}
\setlength{\textheight}{\paperheight}
\addtolength{\textheight}{-4cm}
\addtolength{\textheight}{-1.1\headheight}
\addtolength{\textheight}{-\headsep}
\addtolength{\textheight}{-\footskip}
\setlength{\oddsidemargin}{0.5cm}
\setlength{\evensidemargin}{0.5cm}
\def\P{{\mathbb P}}
\def\E{{\mathbb E}}
\def\degmax{d}
\def\pdim{p}
\def\numobs{n}
\let\hat\widehat
\definecolor{blue}{rgb}{0.01,0.01,0.75}
\def\rc#1{{\it\textcolor{blue}{#1}}\smallskip}
\parindent0pt
\parskip12pt

%%%%%%%%%%%%%%%%%%%%%%%%%%%%%%%%%%%%%%%%%%%%%%%%%%%%%%%%%%%%%%%%%%%%%%%%%%
\renewcommand{\baselinestretch}{1.02}

\begin{document}

\vspace*{5pt}

We thank the reviewers for their careful reading of our paper and many
helpful comments. We have carefully addressed all of these comments. The
following describes the revisions made to the manuscript, with 
responses to each of the points made in the reviews.

\subsection*{Associate Editor}

\rc{Below I recapitulate a few
point of the reviewers', along with some questions/comments of my own.
These (as well as other points in the reviews) need to be thoroughly
addressed in the revised manuscript.}

\begin{enumerate}
\item \rc{Boundary flatness condition: much of the paper's analysis hinges on
this condition.  Reviewer 3 raises some concerns about when it holds,
especially for functions that are not necessarily supported on the
unit hypercube.  See also the point below.}

We have added a new Section 3.2 ``Boundary Flatness Examples'' to the
paper to further discuss the boundary flatness condition.
Extending our analysis to densities whose support
is not a hypercube is beyond the scope of our paper, although we think it
is possible. The key ideas behind our analysis do not depend critically
on the fact that the support is a hypercube. The hypercube assumption
is made only for convenience, to minimize technical details.

\item \rc{Intuitively, variable selection is very closely related to
estimating a function (or possibly its partial derivatives) in the
sup-norm.  It is well-known that accurate estimation of functions
(whether in $L^2$ or the sup-norm) with bounded smoothness in dimension
$s$ requires a sample size that grows exponentially in $s$.  For this
reason, the past result of Comminges and Dalalyan (suggesting an
exponential growth for variable selection) is to be expected.
However, in terms of metric entropy, the class of convex functions
behaves roughly like the class of two-smooth functions (e.g., Dryanov
(2009)).  So at first glance, this makes the result of this paper
rather surprising---namely, that one can somehow side-step any
exponential growth in $s$, even though the metric entropy is still
scaling as badly as the case of 2-smooth functions.  I wondered if the
authors could comment and shed insight on this fact, which seems a
little counterintuitive.  What I suspect that it could mean is that
the additive faithfulness (AF) condition, or one of the sufficient
conditions for AF (namely boundary flatness) are somehow very
restrictive.  Could the authors comment on these issues?}

Boundary flatness doesn't artificially make variable selection
easy. Boundary flatness is much more general than the product density,
and Comminges and Dalalyan showed that even under a product density
variable selection for general two-smooth functions is hard. Any
bounded density with compact support can be approximated arbitrary
well by a boundary flat density (see the newly added Example 3.2).

The intuition behind our result is the following. Start with the fact
that convex functions are additively faithful under a product
density. Next observe that the behavior of a convex function is constrained
by its behavior at the boundary; in particular, it is identically zero
if it's zero on the boundary. Therefore, it is sufficient if the 
underlying density resembles a product density at the boundary.  This is
exactly the notion that boundary flatness formalizes.

At a high level, our results show that for functions with
shape-constraints such as convexity, the metric entropy alone does not
determine the hardness of variable selection.  The shape constraints
also play a direct role.

We have expanded the paragraph discussing the intuition behind
additive faithfulness in the paper. Please see the second paragraph
after Theorem 3.1.

\item \rc{Writing and proofs need to be tidied up in places to make it easier
to follow.  See various comments of referees, particularly the
detailed ones of R2.  Also, in Theorem 3, please remind the author of
when/where the AC/DC estimators were defined.  (I was forced to dig
back through.)}


We have corrected all the minor mistakes found by the reviewers.
We have added a concise definition of AC/DC in Section 2.2, before Theorem 3. 


\end{enumerate}

\subsection*{Reviewer 1}

\rc{Conceptually, I have a few questions for the authors to address:}

\begin{enumerate}[(1)]
\item \rc{If the true regression function $f_0$ can actually be written as a
sum of univariate convex functions (plus noise), are there natural
conditions that one can provide for when additive faithfulness will
hold? Similarly, is there a simpler version of Theorem 3 that can be
stated when $f_0$ is actually a sum of univariate convex functions? This
would seem to depend on an appropriate statement for Assumption (A3).}

If the true regression function is a convex additive function, then
additive faithfulness holds---the best additive approximation
to an additive function is the function itself. The estimation
procedure becomes simpler.  In this case the DC stage is no longer necessary and can
be safely removed.

In the finite sample analysis, the assumptions would not change except that A3 would be $\| f^*_k \|_\infty \leq B$ for all $k$ since $f_0 = f^*$ in this case. The signal strength definition can be made simpler: $\alpha^+ = \min_k \mathbb{E}( f^*_k(X_k)^2 )$. The rates would not improve.


\item \rc{If we actually wish to fit smooth convex univariate components, is
it easy to add this condition as a regularization term and incorporate
it into the analysis?}

Smoothness can be incorporated by using a B-spline basis with uniformly
spaced knots. The details are described in a 2014 paper by Pya and
Wood which we have cited in the introduction. We choose not to add
smoothness into the estimation because it introduces 
additional tuning parameters, which are difficult to set for
additive models.

The analysis would require much more work if one were to use a B-spline basis
in the estimation. In particular, B-splines may complicate the false positive
analysis, which relies on the particular form of the optimization
objective. The false negative analysis may however not be affected much,
because it is mainly based on bracketing entropy.

\item \rc{Assuming model selection consistency, can one derive prediction
error bounds using the fitted $f_k$s and $g_k$s, as well? Prediction error
bounds seem quite relevant due to possible model misspecification when
$f_0$ is not actually a sum of univariate convex functions.}

To minimize predictive error, we can refit a non-additive convex
function after selecting a sparse model. Comparing the predictive
accuracy of the $f^*_k$s and $g^*_k$s against the non-additive true
regression function $f_0$ is beyond the scope of our paper, but we do
compare the finite sample estimate $\hat{f}_k$ and $\hat{g}_k$ against
the population level $f^*_k$ and $g^*_k$. Our analysis to
control false negatives also shows that $$\left| \mathbb{E}\left(
f_0(X) - \sum_{k=1}^p f^*_k(X_k) \right)^2 - \mathbb{E} \left( f_0(X)
- \sum_{k=1}^p \hat{f}_k(X_k))\right)^2 \right| \rightarrow 0$$ and
likewise for the $\hat{g}_k$s. Please see Theorem 8.3 and 8.4 in the
supplement.

We have added a paragraph in Section 3.4 that discusses prediction. 

\item \rc{In order to improve clarity, the authors should define the convex
additive model at the top of page 5: $f(x_i) = \sum_{k=1}^p f_k(x_{ik})$, where the
$f_k$s are convex.  Otherwise, the notation in equation (2.3) is
initially a bit confusing.}

We have removed the old equation (2.3), and moved the discussion of convex
additive models to after AC/DC has been introduced, on page 6.

\item \rc{Another notational comment: the use of the
distribution function $F$ seems unnecessary (e.g., Lemma 3.1 and
Corollary 3.1), since everything is already defined in terms of the
joint density function $p(x)$. (An additional suggestion would be to use
a different letter to represent the joint density function, since $p$ is
already used to represent the dimensionality of the distribution.)}

We sometimes prefer notation involving the distribution because we want to
write, for example, $L^2(P)$ to represent the Hilbert space
of square integrable functions with respect to the
distribution $P$. We have removed any mention of the distribution
where it is not necessary.

We have changed all $F$ notation to $P$ notation to represent the
distribution. To avoid confusing the density with the
dimension, we now write $p(x)$ instead of just $p$ whenever we
refer to the density. We believe that this change, along with the
context, should be enough to disambiguate the two quantities.

\item \rc{I would suggest including the dependence on the positive-valued lower
bound on $p(x)$ (as assumed in (A5)) in the statement of Theorem 3. This
is a point worth commenting upon, since the earlier theorems only
require $p(x)$ to be strictly positive. Assumption (A5) seems like it
could also be interpreted as a sort of minimum signal strength
requirement for successful variable selection.}

We have added the dependence on the lower bound $c_l$ of $p(x)$ in
Theorem 5.2, where we require that $p \leq O(\exp(cn))$ for $c <
\frac{c_l}{32}$. We have also added the dependence on the upper bound
$c_u$ of the density in Theorem 5.3. The lower and upper bounds $c_l,
c_u$ do not directly play the role of the signal level; they are used
to simplify technicalities in the presentation and the proof of the
theorems. For example, Theorem 5.2 should also go through if one
assumes only that the marginal densities $p(x_j)$ are positive on a
sub-interval of $[-1,1]$.

\item \rc{It would be good to point to Assumption (A3) at the beginning of
Section 5 when discussing the $B$-boundedness condition. In the third
paragraph, the authors might note that the $B$-boundedness condition
arises from the analysis of Theorem 5.2. To clarify, the penultimate
paragraph on page 22 should state that Theorem 5.1 controls false
positives alone, whereas Theorem 5.3 (to follow) provides
finite-sample results controlling the false negative rate.}

We have made the suggested modifications except for the one in the second sentence. The $B$-boundedness condition is used in both Theorem 5.2 and Theorem 5.3. It is relied upon critically in Theorem 5.3 and used for convenience in Theorem 5.2. 


\item \rc{At the top
of page 23, the remark regarding mutual incoherence conditions is a
bit confusing. Aren’t the conditions imposed in Theorem 5.1 on the
covariates essentially analogous to the mutual incoherence conditions
in this case? }

The conditions in Theorem 5.1 are indeed analogous and we have edited the remark in the paper accordingly. The conditions in Theorem 5.1 are much more strict than the mutual incoherence condition however. 

\item \rc{In the simulation section, the authors might consider comparing the
set of variables selected after the AC step only with the set of
variables selected via the entire AC/DC algorithm. This would help
illustrate the necessity of the concave fitting step in addition to
the convex additive fit alone.}

We have added an additional simulation comparing the variables
selected via the AC stage and the DC stage. Please see the last
paragraph in Section 6.1 as well as Figure 6. The DC stage is slightly
helpful in recovering the true variables but its effect is
insignificant. We believe that the DC stage is important in theory, but
not in practice.

Minor typos:
\begin{enumerate}[(i)]
\item \rc{In Section 2.3, it would be good to actually state the regularized
objective function being analyzed, rather than waiting until Section
4.}

We have added the objective function in Section 2.2 where we summarize the optimization algorithm.

\item \rc{Assumption (A3) on page 7 is missing the subscript $\infty$.}

The missing subscript is added.

\item \rc{The parameter $\sigma$ is never defined in Section 2.3. Reading ahead,
it seems to be the sub-Gaussian parameter of the additive noise.}

$\sigma$ has been defined in Section 2.3.

\item \rc{Perhaps Section 3 should be titled something other than ``Additive
faithfulness'' (maybe ``Population- level results''?).}

We have changed the title of Section 3 to the suggested one.

\item \rc{The first paragraph of Section 5 should read
  ``Figure 3'' rather than ``Algorithm 3.''}

Corrected.

\item \rc{The second paragraph of Section 5 has $m\hat{}u$ instead of $\hat\mu$.}

Corrected.

\item \rc{The first sentence of Section 5.1 should read optimization (4.8).}

Corrected.

\end{enumerate}

\end{enumerate}

\subsection*{Reviewer 2}


We are grateful to the reviewer for the detailed reading. We have corrected the mistakes found and further proof-read the paper. We discuss the nontrivial issues here.  

\rc{I found many (unimportant) mistakes in the proofs (I
  did not mention them all in the minor comments), which need to be
  read again carefully.}

\begin{enumerate}
\item \rc{I have a problem with the assertion of
  uniqueness in Lemma 3.1 and Theorem 3.2. First, in Lemma 3.1, it is
  clear that $f^∗$ is unique. But $(f_1^∗, \ldots, f_p^∗)$ is not. What is
  shown is that, for $k$ fixed and given $(f_j^*)_{j\neq k}$, $f_k^*$ is unique. Next,
  in Theorem 3.2, the proof of uniqueness (bottom of page 16 and top
  of page 17) is not at all correct.
   Now, to see why I think that, with the assumptions made on $X$, there is
   not uniqueness, simply suppose for instance that $p = 2$ and $X_2 = g(X_1)$
   for a certain function $g$. Then if $(f_1^∗, f_2^∗)$ is a solution,
   $(f_1^∗ + f_2^* \circ g, 0)$ is also a solution.}


We have corrected the proof of Lemma 3.1 and we have given a more
rigorous proof that the component $f^*_k$s are unique in a new Lemma
8.3 (old Lemma 8.3 was removed). Our result requires that the density
is strictly positive, and thus, the $X_2 = g(X_1)$ example raised by the
reviewer is ruled out.

%\item \rc{In the definition of $g_k^∗$ on pages 6 and 15, it would be more precise to
%write $g_k(X_k)$ in the sum instead of $g_k$.}

%\item \rc{On page 9, in the proof of Lemma 3.1, the star has been forgotten
%twice on $f_{k'}$ in (3.2) and (3.7).}

%\item \rc{On page 10, on line 3, ``therefore the solution $f_k^∗(x_k) = \E[f(X)|x_k]
%− \E f(X)$ is unique'', the term $−\sum_{k'\neq k} f_{k'}(X_{k'})$ has been forgotten.}

%\item \rc{On page 10, on line 14, ``In particular, if $F$ is the uniform
%distribution etc'', the term $− \int f$ has been forgotten.}

%\item \rc{On page 10, on line 16, ``additively'' should be replaced by
%``additive''.}

%\item \rc{On page 11, at the bottom of the page, ``before we presenting''}

\item \rc{On page 11, after Definition 3.2, it is written ``when the joint
density satisfies the property that
$\frac{\partial(x_{-j},x_j)}{\partial x_j} = \frac{\partial^2 p(x_{-j},
  x_j)}{\partial x_j^2} = 0$ at boundary points''.
I think the condition ``$p(x_{−j}, x_j ) = 0$ at the boundary points'' is
missing.}

It is necessary that $p(\mathbf{x}_{-j}, x_j) > 0$. We have added an additional condition that $p(\mathbf{x})$ has a bounded second derivative. This condition ensures that the marginal density $p(x_j)$ is twice differentiable and $p'(x_j) = p''(x_j) = 0$ at $x_j = 0,1$. With this, it is valid to apply the quotient rule for differentiation to show that boundary flatness holds. Please see Example 3.2 for the relevant edits.


\item \rc{On page 12, I did not understand why the functions 
$\frac{\partial^2 p(x_{-j} |  x_j)}{\partial x_j^2}$
and 
$\frac{\partial^2 f(x_{-j} |  x_j)}{\partial x_j^2}$
are bounded (they are not supposed to be continuous)}


We do need a boundedness condition here. We have modified the
definition of boundary flatness so that $p(\mathbf{x_{-j}} \,|\, x_j)$
is bounded and has a bounded first and second derivative for all $x_j$
in some open set around the boundary points and for all
$\mathbf{x}_{-j}$. The dominated convergence theorem holds 
because we only exchange the derivative and the integral at $x_j =
0,1$. We have verified in detail that the domination condition holds
in the newly added Section 8.4.1.

%\item \rc{On page 13, on line 6, ``this this''.}

\item \rc{On page 13, on line 23, I do not think $\Sigma$ has been defined. Besides it
is not specified that $Var(X_1) = 1$.}

$\Sigma$ has been defined as the covariance matrix with diagonal entries equal to 1.

%\item \rc{On page 13, on line 29, ``additive'' should be replaced by
%``additively''.}

\item \rc{On page 13, the assumption $\E f(X) = 0$ should be made, or else the
parameter $\mu$ should be introduced as before.}

We have added the $\E f(X) = 0$ assumption.

%\item \rc{On page 16, in the expression of $c^∗$, the
%  denominator should be $\E X_k^2 −m_k^2$.}

\item \rc{On page 16, in the proof of Theorem 3.2, Lemma 8.3 is used. This
lemma supposes that $\phi$ is continuous. Here I do not see why $\phi =
\sum_{k'\neq k} f_{k'}*$ is continuous (it is convex then continuous on the interior of
$[0,1]^p$. ).}

We have updated the statement and the proof of Theorem 3.2 so that only boundedness is required of $\phi$. Furthermore, $h^*_k$ is no longer required to be twice differentiable; it need only be square-integrable. The old Lemma 8.3 is no longer necessary and has been removed. 

\item \rc{On page 17, same problem as before with the absence of parameter $\mu$.}

$\mu$ has been added.

%\item \rc{On page 17, on line 16 ``for that variable. for each''}

%\item \rc{On page 18, at the bottom of the page :
%  ``$\beta_{\pi_k(i+1)k} 
%\geq \beta_{\pi_k(i)k}$ for $i=1,\ldots, n-1$'' , $n−1$ must be replaced by $n−2$.}

\item \rc{On page 19, $\sum_{i} f_{ik}$ instead of $\sum_i
  f_{ki}$.}

This has been fixed. We have also gone through the paper and corrected
all instances of the $f_{ik}, f_{ki}$ notation inconsistency so that
$i$ always comes before $k$.

%\item \rc{On page 20, in the third paragraph, on the second, third and fourth
%lines, $j$ and $j − 1$ must be replaced by $j + 1$ and $j$ respectively in
%$x_{\pi_k(j)k}$ and $x_{\pi_k(j-1)k}$.}

%\item \rc{On page 20, In the central matrix equation, the
%  first vector should be
%  $\left[\begin{matrix}f_k(x_{1k}\\ f_k(x_{2k}\\ \vdots\\f_k(x_{nk}\end{matrix}\right]$. N%ext
%  in ``$\mu_k = -\frac{1}{n} 1_n^T \Delta_k d_k$``, $1_n$ is missing.}
%\item \rc{On page 21, in Section 5, at the end of the
%  second paragraph, ``$m\hat{}u$ from
%our estimation procedure''.}


%\item \rc{On page 22, in Definition 5.1, the inequalities
%  about $d_k$ must be replaced by the same inequalities as in Proposition 4.1.}

\item \rc{On page 22 (and on page 31), in the statement of Theorem
  5.1, in\\ $\max_{i=1,\ldots, n} \frac{X_{k\pi_k(i+1)}-
    X_{k\pi_k(i)}}{\text{range}_k} \geq \frac{1}{16}$ the
sign $\geq$ must be replaced by $\leq$ . Besides I think it
is a bad idea to change notation : $X_{k\pi_k(i)}$ should
be replaced by $X_{\pi_k(i)k}$ as before (page 19 for instance).}

We have corrected all instances of this notation inconsistency.

%\item \rc{On page 23, in equation (5.1), in ``$f^∗(X)$'', the star must be
%removed.}

\item \rc{On page 24, ``$\alpha_{+} > 0$ since $f$ is the unique risk minimizer'' : I do
not understand this statement (there is an infimum, not a minimum in
the definition of $\alpha_{+}$.)}

The infimum is achieved by some $f \neq f^*$ since the constraint set
$\{ f \in \mathcal{C}^p_1 \,:\, \textrm{supp}(f) \subsetneq
\textrm{supp}(f^*)\}$ is a finite union of closed convex sets. It is
$\bigcup_k \{ f \in \mathcal{C}^p_1 \,:\, \textrm{supp}(f) =
\textrm{supp}(f^*) - \{k\}\}$. Therefore, the infimum is the minimum
of a finite number of projections onto closed convex sets. We have
changed the ``inf'' to ``min'' and added an explanation in the paper
to remove the confusion.

%\item \rc{On page 24, in Remark 5.1, ``In important direction''.}

\item \rc{On page 25, in the statement of Theorem 5.3, the square must be
removed on $\alpha_-$.}

The square is not an error. We have a stronger requirement on the
signal strength for the DC stage. This is because the $\hat{g}_k$s
are fitted onto the finite sample residual and thus, we need $\hat{f}$
to be close to $f^*$ in order to be able to prove something about the
$\hat{g}_k$s.

\item \rc{On page 25, after the statement of Corollary 5.1 : I do not understand
why ``$p = O(exp(n^c)$'' is written instead of ``$p = O(exp(cn)$''.}

We have changed it to $\exp(cn)$.

%\item \rc{On page 26, in the first paragraph ``the relevant variable set'' and
%in the third paragraph, ``seting''.}

\item \rc{I really did not understand what was represented in figures 5(a) and
5(b) (axis?). Besides it does not seem to correspond with what is
described at the bottom of page 26 and at the top of page 27. The
notation $\|\cdot\|_{\infty,1}$ is not defined.}

We have further explained the old Figures 5a and 5b (now Figure 7a and
7b) in the experiment section. The following has been added to the
paper:

``... we plot on the $y$-axis the norm $\|f^{(t)}_j\|_{\infty}$ of every
column $j$ against the regularization strength
$\lambda^{(t)}$. Instead of plotting the value of $\lambda^{(t)}$ on
the $x$-axis however, we plot the total norm at $\lambda^{(t)}$
normalized against the total norm at $\lambda^{(1)}$: $\frac{\sum_j
  \|f^{(t)}_j\|_{\infty}}{\sum_j \|f^{(1)}\|_{\infty}}$. Thus, as
$x$ moves from 0 to 1, the regularization goes from strong to
weak.''

\item \rc{On page 31, at the top of the page, ``see discussion at beginning of
Section 5'' : I think it should be ``at the beginning''. In (8.1),
same problem as before with the indices : ``$\nu_{ki} d_{ki}$ '' must be replaced
with ``$\nu_{ik}d_{ik}$'' (see for instance page 20). At the bottom of the page,
$v_{ki}$ must be replaced by $\nu_{ki}$.} 

The typo and the index notation inconsistency have been corrected.

\item \rc{On page 32, on the first line of the second paragraph, ``$d_k = 0 \mbox{for $k
\in S$}$'', $S$ must be replaced by $S^c$ and $\|\mu\|$ by $\|\mu\|_1$ and ``It clear
that''. Next, the third paragraph (``to ease notational ....'') should
be placed at the beginning of the proof (it is used from the beginning
of the proof, cf conditions on $d_k$).}

We cannot move the third paragraph to the beginning because it is not
clear at that point that $k$ is fixed. We have corrected all of the
notation before the third paragraph to be
consistent.

\item \rc{On page 33, I find that $[\lambda\Delta_k^Tu]_1 = \lambda((X_{kn} −X_{k1})\kappa$. ( If $\kappa$ is be defined
in the following way : $\kappa = \frac{1}{\lambda_n(X_{kn}-X_{k1})}[\Delta_k^T u]_1$, then everything
goes well).}

The original definition of $\kappa$ was indeed wrong; we forgot to update $\kappa$ when we edited the proof. We have corrected this in the paper.

%\item \rc{On page 33, on the second line, ``other rows of stationarity
%condition holds'' and in the third paragraph : ``following our strategy, We''.}

%\item \rc{On page 34, at the bottom of the page, the parentheses in the
%denominator of the fraction.}

\item \rc{On page 35, in the third paragraph, ``as a function of $\nu$'' : it is
$\zeta$, not $\nu$. In the fourth paragraph, $\nu$ is not defined. In the fifth
paragraph, in the equation, remove the sum $\sum_{k=1}^p
\bar\Delta_k^T u_k$  and replace
it by the vector ̄ˆ with componen$(\bar\Delta_k^T u_k)_{k=1,\ldots, p}$ In the sixth paragraph,
in $\bar\Delta \hat d_k = \bar\Delta\hat d$ replace $\hat d_k$ by
$\hat d'$.}

We have corrected the expression. The sum has been replaced by a vector.

\item \rc{On page 36, maybe it would be better to replace $d$ by $c$ here. Add
``for $k\in S^c$ after''to show that $\hat c_k =0$''.}

We have replaced $d$ by $c$. 

\item \rc{On page 37, In the first paragraph, I do not understand why $p$ appears
here. In the last paragraph, I find 4 instead of 2 on the first term of
the second line (and then 12 instead of 8 on the third line) .}

We have removed mentions of $p$ here and corrected the constants.

%\item \rc{on page 38, In the sixth paragraph, in the inequality, the sign
%should be $\leq$. In the eighth paragraph, $c''$ must be replaced by $c$. In
%the ninth paragraph, ``Taking an union''.}

%\item \rc{On page 40, on the fifth line, ``empricial''. In the last paragraph,
%``whose size is bounded'' : it is the log of the size.}


\item \rc{I had a problem with the definition of an ε-bracketing. First, in the
definition 8.1, on page 49, `` we define a bracketing'' should be
replaced by `` we define a $\epsilon$-bracketing''. Then it is said that $\rho(f_L ,
f_U ) \leq \epsilon$ and that the corresponding size is $N(\epsilon,C,\rho)$. But then, in
Proposition 8.3 (and almost everywhere else), when $N(2\epsilon,L_2(P),\rho)$ is
used, it is written $\|f_L − f_U \|_{L_2} \leq \epsilon$ (instead of $2\epsilon$. ) (same thing in
Corollary 8.4 for instance : $\epsilon$-bracketing with a size $N(2\epsilon,...)$)}

We have corrected this problem by replacing all instances of $N(2\epsilon,...)$ with $N(\epsilon, ...)$ and updating the bound for $\log N(...)$ accordingly. 

%\item \rc{On page 41, problem with $\epsilon$ (same as the previous item).}
\item \rc{On page 40, I think the functions in $G$ are bounded by $2sB$ (not $sB$.).}

This has been noted and corrected across the paper.

\item \rc{On page 49, in Definition 8.1, $f^U$ and $f^L$ must be replaced by $f_U$ and
$f_L$. On the last line of the page, ``additive convex functions with $s$
components'' : add ``components bounded by $B$''.}

We have corrected the instances in which we erroneously omitted ``...components bounded by $B$''.

\item \rc{On page 49, ``$\int p(x)^2dx \leq (\int p(x)dx)^2$ ``??}

We have corrected this mistake by placing an upper bound $c_u$ on the density $p(\mathbf{x})$ and incorporating that upper bound into this derivation. We now use the correct bound $\sqrt{ \int p(x)^2 dx } \leq c_u$.

\item \rc{On page 50, a constant 2 is missing in $\epsilon_n$.}

This mistake has been corrected across the paper.

\item \rc{On page 41, in the second paragraph, I think the bound on $\sup_{h_L} |⟨w,h_L⟩_n|$ is
wrong. I would have used the fact that the variables $h_L(X_i) W_i$ are
independent, centered and sub-Gaussian with a scale smaller than
$2\sigma s B$. It gives a bound of order $sB\sigma \sqrt{\frac{\log\frac{2}{\delta}}{n}}$.}

We have updated this part of the proof.

\item \rc{The term $\epsilon$ supposed to balance the two terms is not correct (for
instance it does not contain $\delta$). There are many small mistakes in the
calculations. In particular, the exponent in $s$ is not the right one.}

The $\epsilon$ we choose is sufficient to get the bounds we want but
it does not exactly balance the terms. We left dependence on $\delta$
out of the $\epsilon$ intentionally because we need to verify that
$\epsilon$ satisfies certain conditions; see Equation (8.12) for
instance. These conditions are much easier to verify if $\epsilon$
does not depend on $\delta$, which we set to be $1/n$ only at the
end. We have updated the paper to clearly state that we use a
suboptimal $\epsilon$.

\item \rc{In the proof of Theorem 8.4 : too many small mistakes in
calculations, but the final result is OK.}

We have carefully proof-read and edited the proof of Theorem 8.4. 

\item \rc{On page 45, the term $\epsilon$ supposed to balance the terms is not correct
(same remark about $\delta$). In the last equations, on the first line (bottom
of the page) $n$ is missing in one of the scalar products.}

We also use a suboptimal $\epsilon$ here for convenience. We have updated the paper to clearly state this.

%\item \rc{On page 46: ``taking a union bound and we have that''}
\item \rc{In Lemma 8.3, I do not see why the functions are continuous : the
functions $x_{-k} \rightarrow p(x_{-k} | x_k)$ are not supposed to be continuous .}

The old Lemma 8.3 has been removed. We now just need $x_{-k}
\rightarrow p(x_{-k} \,|\, x_k)$ to be bounded, which has been added
as a condition. (We now assume for convenience that the joint density
$p(\mathbf{x})$ is bounded away from 0 and $\infty$.)

\item \rc{On page 47, in the proof of Lemma 8.4 : notation $\phi_{-j}$
  has not been defined.
In the definition of $A_+$, the sign should be $>$, not $\geq$. Remove ``both'' .}

The old Lemma 8.4 (now Lemma 8.3) has been updated. 

%\item \rc{On page 47, ``a sub-exponential random is''.}

\end{enumerate}

\subsection*{Reviewer 3}

\begin{enumerate}

\item \rc{I don't see why one would use convex regression to do
  variable screening when other parametric approaches seem more
  appropriate (e.g. clean and screen methods by Ke et al. and
  others). This method and others were compared with and so it again
  raised the question of what class of interesting examples would this
  method be useful for.}

Our methods are appropriate in settings where one expects the
regression function to be convex, for example in utility function
estimation in economics. The power of our framework and analysis is that
using convex additive models leads to consistent variable selection.
In more general settings, without this shape constraint,
approaches based on the lasso (``screen and clean'') or nonparametric 
additive models do not have such guarantees. 

Also note that we have replaced the multivariate Gaussian examples with 
nonparametric examples; please see point 3 below.


\item \rc{The basis of all the results seem to be the boundary flatness
condition. This condition seems to be an artifact of the analysis and
based on my reading, it is unclear how restrictive this assumption
is. The authors claim it is weak just after stating it and then
present several straightforward examples in which it doesn't hold
(e.g. $f(x_1,x_2)=x_1 x_2$) and multivariate Gaussian examples. Hence it is unclear to me how
likely a given function is likely to fit into this framework.}

The boundary flatness condition is not restrictive in the sense that
for any density $p(x)$ supported on $[0,1]^p$, and for any $\epsilon >
0$, there exists a boundary flat density whose $L_1$ distance with
$p(x)$ is at most $\epsilon$. Please see the newly added Example
3.3. This is because boundary flatness places no restriction on what
$p(x)$ looks like in the interior of the hypercube. Boundary flatness
allows arbitrary correlation structure, provided that $p(x) > 0$. We
have added Section 3.2 to the paper to give a more detailed discussion
of boundary flatness.

$f(x_1,x_2) = x_1 x_2$ is non-convex and it is an example of how
non-convex functions may not be additively faithful under the uniform
distribution; it is not an example for which boundary flatness fails
to hold. The multivariate Gaussian counterexample is used to show
that boundary flatness may not extend easily to densities with
unbounded support.

We acknowledge that many distributions are not boundary flat. We
present boundary flatness as a significant generalization of the
product density, under which nonparametric variable selection is
already very difficult (Comminges and Dalalyan, 2012).

Boundary flatness is not a contrived condition. We initially proved
additive faithfulness only for product densities. Then, knowing that
the behavior of a convex function everywhere is constrained by its
behavior at the boundary, we conjectured that the underlying density may
only need to ``look like'' a product density at the boundary. Attempts
to formalize this notion led us to the definition of boundary
flatness.

\item \rc{Following on from the previous point, the simulation study also does
not add any further insight to this issue since the simulations are
for multivariate Gaussians where other parametric approaches that deal
with covariance seem more suitable. It is puzzling to me that the
authors did not consider non-parametric examples where the additive
model framework seems more suitable.}

We have replaced the experiments that use correlated multivariate
Gaussians with new experiments that use
boundary flat Gaussian Copula mixture distributions. Please see
Section 6.1; Figure 4 in particular shows the bimodal marginal density
of the new joint density that we use. Figures 5d, 5e, and 5f show the
result of simulations using the new non-Gaussian density.

We have also added new experiments that use nonparametric 
regression functions instead of the quadratic form. We use a
log-sum-exp (softmax) of linear forms (please see Equation
6.1). Figures 5f and 6 show the results of experiments using the
softmax as the true regression function: AC/DC is still effective
under a high-dimensional scaling.

\end{enumerate}

Minor comments:

\begin{enumerate}
\item \rc{I didn't think there was any need for Section 2
  since it feels very repetitive and slightly confusing that parts of
  the text are repeated almost verbatim. I would suggest removing this section.}

Section 2 is repetitive but we believe it has value in highlighting
the big picture, which could otherwise be obscured by the technical
details.

\item \rc{Page 2, second last paragraph `and and.'}

We have corrected this.

\end{enumerate}


\vspace*{10pt}

Sincerely, 


Min Xu, Minhua Chen, and John Lafferty\\[1pt]
\today{}

\bibliography{local}
\end{document}
