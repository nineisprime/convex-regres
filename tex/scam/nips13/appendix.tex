\section*{Appendix}

\subsection{Proof of Additive Faithfulness}
\label{sec:faithful_proof}

We start with a lemma similar, and whose proof is identical to Lemma~\ref{lem:general_int_reduction}.

\begin{lemma}
\label{lem:int_reduction}
Let $\mu$ be a probability measure on $C=[0,1]^s$. Let $f: C \rightarrow \R$ be a convex function, twice differentiable. Suppose that $\E f(X) = 0$.

Let $\ds f^*_1,...,f^*_s \coloneqq \arg\min \{ \E | f(X) - \sum_{k=1}^s f_k(X_k) |^2 \,:\, \forall k, f_k \trm{ convex }, \E f_k(X_k) = 0\}$ 

Then $f^*_k(x_k) = \E[ f(X) | x_k]$.
\end{lemma}
 
 \begin{proof}
 
Let $f^*_1,...,f^*_s$ be the minimizers as defined. It must be then that $f^*_k$ minimizes $\{ \E | f(X) - \sum_{k' \neq k} f^*_{k'} (X_{k'}) - f_k (X_k)|^2 \,:\, f_k \trm{ convex }, \E f_k(X_k) = 0 \}$.

Fix $x_k$, we will show that the value $\E[ f(X) | x_k]$ minimizes 
\[
\min_{ f_k(x_k) } \int_{\mathbf{x}_{-k}} p(x) | f(\mathbf{x}) - \sum_{k' \neq k} f^*_{k'} (x_{k'}) - f_k (x_k)|^2 d \mathbf{x}_{-k}.
\]

Take the derivative with respect to $f_k(x_k)$ and set it equal to zero, we get that
\begin{align*}
\int_{\mathbf{x}_{-k}} p(\mathbf{x}) f_k(x_k) d \mathbf{x}_{-k} &= \int_{\mathbf{x}_{-k}} p(\mathbf{x}) ( f(\mathbf{x}) - \sum_{k' \neq k} f^*_{k'} (x_{k'}) ) d \mathbf{x}_{-k}  
p(x_k) f_k(x_k) &= \int_{\mathbf{x}_{-k}} p(\mathbf{x}) f(\mathbf{x}) d \mathbf{x}_{-k} 
 \end{align*}

Now, we verify that as a function of $x_k$, $\E[ f(X) | x_k]$ has mean zero and is convex. The former is true because $\E f(X) = 0$; the latter is true because for every $\mathbf{x}_{-k}$, $f(x_k, \mathbf{x}_{-k})$ is a convex function with respect to $x_k$ and therefore, $\int_{x_{-k}} p(\mathbf{x}|x_k) f(x_k, \mathbf{x}_{-k}) d \mathbf{x}_{-k}$ is still convex. 
 
 \end{proof}
 
 The next proposition, combined with the previous Lemma, proves Theorem~\ref{thm:convex_faithful}.
 
 \begin{proposition}
Let $p$ be a product probability distribution supported and positive on $C=[0,1]^s$ so that $X_1,...,X_s$ are all independent. Let $f: C \rightarrow \R$ be a convex function, twice differentiable. 

Let $f^*_1,...,f^*_s \coloneqq \arg\min \{ \E | f(X) - \sum_k f_k(X_k) |^2 \,:\, \forall k, f_k \trm{ convex }, \E f_k(X_k) = 0\}$.

The following are equivalent:
\begin{packed_enum}
\item $f$ does not depends on coordinate $k$
\item For all $x_k$, $\E[ f(X) | x_k] = 0$.
\end{packed_enum}
\end{proposition}

\begin{proof}

The first condition trivially implies the second because $\E f(X) = 0$.

Fix $k$. Suppose that, for all $x_k$, $\E[ f(X) | x_k] = 0$. 

By the assumption that $p$ is a product measure, we know that, for all $x_k$,
\begin{align*}
p(x_k) \E[ f(X) | x_k ] &= \int_{\mathbf{x}_{-k}} p(\mathbf{x_{-k}} \,|\, x_k) f(x_k,\mathbf{x}_{-k}) d \mathbf{x}_{-k} \\
		&=  \int_{\mathbf{x}_{-k}} p(\mathbf{x_{-k}}) f(x_k,\mathbf{x}_{-k}) d \mathbf{x}_{-k} =0
\end{align*}

For every $\mathbf{x}_{-k}$, we define the derivative
\[
g(\mathbf{x}_{-k}) \coloneqq  \lim_{x_k \rightarrow 0^+} \frac{f(x_k, \mathbf{x}_{-k}) - f(0, \mathbf{x}_{-k})}{x_k}
\]
$g(\mathbf{x}_{-k})$ is well-defined by the assumption that $f$ is everywhere differentiable.

We now describe two facts about $g$.

Fact 1. By exchanging limit with the integral, which is valid by bounded convergence theorem, we reason that 
\[
\int_{\mathbf{x}_{-k}} p(\mathbf{x}_{-k}) g(\mathbf{x}_{-k}) d \mathbf{x}_{-k} = 0
\]

Fact 2. Because $f$ is convex, $g(\mathbf{x}_{-k})$ is a component of the subgradient $\partial_{\mathbf{x}} f(0, \mathbf{x}_{-k})$. (the subgradient coincides with the gradient by assumption that $f$ is twice differentiable)

Therefore, using the first order characterization of a convex function, we have
\begin{align*}
f(\mathbf{x}') &\geq f( \mathbf{x}) + \partial_{\mathbf{x}} f(\mathbf{x})^\tran (\mathbf{x}' - \mathbf{x}) \quad \trm{ for all } \mathbf{x}', \mathbf{x} \\
f(x_k, \mathbf{x}_{-k}) &\geq f(0, \mathbf{x}_{-k}) + g(\mathbf{x}_{-k}) x_k \quad \trm{ for all } x_k, \mathbf{x}_{-k}
\end{align*}


For all $x_k, \mathbf{x}_{-k}$, 
\[
f(x_k, \mathbf{x}_{-k}) - f(0, \mathbf{x}_{-k}) - g(\mathbf{x}_{-k}) x_k \geq 0
\]
and
\[
\int_{\mathbf{x}_{-k}} p(\mathbf{x}_{-k}) (f(x_k, \mathbf{x}_{-k}) - f(0, \mathbf{x}_{-k}) - g(\mathbf{x}_{-k}) x_k) d\mathbf{x}_{-k} = 0
\]
Since $f(x_k, \mathbf{x}_{-k}) - f(0, \mathbf{x}_{-k}) - g(\mathbf{x}_{-k}) x_k$ is a continuous function of $\mathbf{x}_{-k}$ and the density $p(\mathbf{x}_{-k})$ is, by assumption, non-zero on the support, we conclude that for all $x_k, f(x_k, \mathbf{x}_{-k}) - f(0, \mathbf{x}_{-k}) - g(\mathbf{x}_{-k}) x_k =0 $ necessarily and thus $\mathbf{x}_{-k}$, $f(x_k, \mathbf{x}_{-k}) = f(0, \mathbf{x}_{-k}) + g(\mathbf{x}_{-k}) x_k$.

The Hessian of $f$ then (guaranteed to exist by assumption) has a zero on the $k$-th main diagonal entry. 

By proposition 7.1.10 from Horn and Johnson~\cite{HJ90},  such a matrix is positive semidefinite if and only if the $k$-th row and column are also zero. (to prove this, one can first work with a 2 by 2 matrix and then use the fact that every principal submatrix of a PSD matrix must be PSD) 

Since $k$-th row and column correspond precisely to the gradient of $g(\mathbf{x}_{-k})$, we conclude that $g$ must be a constant function. It follows therefore that $g = 0$ because it integrates to 0. 

So we have that for all $x_k, \mathbf{x}_{-k}$, $f(x_k, \mathbf{x}_{-k}) = f(0, \mathbf{x}_{-k})$, which concludes our proof.

\end{proof}
 
 
 
 
 
 
 
 
 
 \subsection{Proof of the Deterministic Condition for Sparsistency}
 \label{sec:deterministic_proof}
 
 We restate Theorem~\ref{thm:deterministic} first for convenience.
 
\begin{theorem} 
The following holds regardless of whether we impose the boundness and smoothness condition in optimization~\ref{opt:alternate_opt} or not.

For $k \in \{1,...,p\}$, let $\Delta_{k, j}$ denote the $n$-dimensional vector $\max( X_k - X_{k (j)} \mathbf{1}, 0)$. 

Let $\{\hat{d}_k, \hat{c}_k \}_{k \in S}$ be the minimizer of the restricted regression optimization program~\ref{opt:alternate_opt}. 
Let $\hat{d}_k = 0$ and $ \hat{c}_k = 0$ for $k \in S^c$.

Let $\hat{r} \coloneqq Y - \sum_{k \in S} (\Delta_k \hat{d}_k - \hat{c}_k \mathbf{1})$ be the residue.

Suppose for all $j,k$, $\lambda_n > | \frac{1}{n} \hat{r}^\tran \Delta_{k,j}|$, then $\hat{d}_k, \hat{c}_k$ for $k=1,...,p$ is an optimal solution to the full regression \ref{opt:alternate_opt}.

Furthermore, any solution to the optimization program \ref{opt:alternate_opt} must be zero on $S^c$.
\end{theorem}

\begin{proof}
We will omit the boundness and smoothness constraints in our proof here. It is easy to add those in and check that the result of the theorem still holds.

We will show that with $\hat{d}_k, \hat{c}_k$ as constructed, we can set the dual variables to satisfy complementary slackness and stationary conditions: $\nabla_{d_k, c_k} L(\hat{d})  = 0$ for all $k$.

we can re-write the Lagrangian $L$, in term of just $d_k,c_k$, as the following.
\[
\min_{d_k, c_k}  \frac{1}{2n} \| r_k - \Delta_k d_k + c_k \mathbf{1} \|_2^2 + \lambda \sum_{i=2}^n d_{ki} + \lambda|d_{k1}| - \mu_k^\tran d_k + \gamma_k (c_k - \mathbf{1}^\tran\Delta_k d_k)
\]
where $r_k \coloneqq Y - \sum_{k' \in S, k' \neq k} (\Delta_{k'} d_{k'} - c_{k'} \mathbf{1})$, and $\mu_k \in \R^n$ is a vector of dual variables where $\mu_{k,1} = 0$ and $\mu_{k,i} \geq 0$ for $i=2,...,n$.

First, note that by definition as solution of the restricted regression, for $k \in S$, $\hat{d}_k, \hat{c}_k$ satisfy stationarity with dual variables that satisfy complementary slackness. 

Now, let us fix $k \in S^c$ and prove that $\hat{d}_k = 0, \hat{c}_k=0$ is an optimal solution. 

\begin{align*}
\partial d_k :& \qquad -\frac{1}{n} \Delta_k^\tran ( \hat{r} - \Delta_k \hat{d}_k + \hat{c}_k \mathbf{1}) + \lambda \mathbf{u}_k  - \mu_k - \gamma_k \Delta_k^\tran \mathbf{1} \\
\partial c_k: & \qquad -\frac{1}{n} \mathbf{1}^\tran ( \hat{r} - \Delta_k d_k + c_k \mathbf{1}) + \gamma_k
\end{align*}
In the derivatives, $\mathbf{u}$ is a $(n-1)$-vector whose first coordinate is $\partial | d_{k1}|$ and all other coordinates are 1.

We now substitute in $\hat{d}_k = 0, \hat{c}_k=0$ and show that the duals can be set in a way to ensure that the derivatives are equal to 0.

\begin{align*}
-\frac{1}{n} \Delta_k^\tran\hat{r} + \lambda \mathbf{u} - \mu_k - \gamma_k \Delta_k^\tran \mathbf{1} &= 0 \\
-\frac{1}{n} \mathbf{1}^\tran \hat{r} + \gamma_k & = 0 \\
\end{align*}
where $\mathbf{u}$ is 1 in every coordinate except the first, where it can take any value in $[-1,1]$.

First, we observe that $\gamma_k = 0$ because $\hat{r}$ has empirical mean 0. All we need to prove then is that 
\[
\lambda \mathbf{u} - \mu_k = \frac{1}{n} \Delta_k^\tran \hat{r}.
\]

Suppose
\[
\lambda \mathbf{1} > |\frac{1}{n} \Delta_k^\tran\hat{r}|,
\]
then we easily see that the first coordinate of $\mathbf{u}$ can be set to some value in $(-1,1)$ and we can set $\mu_{k,i} > 0$ for $i=2,...,n$. 

Because we have strict inequality in the above equation, Lemma 1 from ~\cite{Wain:09a} show that all solutions must be zero on $S^c$.
\end{proof}
 
 
 
 
 
 
 \subsection{Proof of False Positive Control}
 \label{sec:false_positive_proof}
 
 We first restate the theorem for convenience.
 
 
\begin{theorem} 
Suppose assumptions A1, A2, A3 hold. 

Suppose $\lambda_n \geq c b (sB + \sigma) \sqrt{ \frac{s}{n} \log n \log (pn)}$, then with probability at least $ 1 - \frac{C}{n}$, for all $j,k$,
\[
\lambda_n >  | \frac{1}{n} \hat{r}^\tran \Delta_{k,j}|
\]
And therefore, the solution to the optimization~\ref{opt:alternate_opt} is zero on $S^c$.
\end{theorem}

\begin{proof}
The key is to note that $\hat{r}$ and $\Delta_{k,j}$ are independent for all $k,j$ because $\hat{r}$ is only dependent on $X_{S}$.

\textbf{Step 1.} We first get a high probability bound on $\| \hat{r} \|_\infty$. 

\begin{align*}
\hat{r}_i &= Y_i - \sum_{k \in S} \hat{f}_k (X_k^{(i)}) \\
	&= f^*(X_S^{(i)}) + \epsilon_i - \bar{f^*} - \bar{\epsilon}- \sum_{k \in S} \hat{f}_k(X_k^{(i)}) \\
	&=  f^*(X_S^{(i)}) - \bar{f^*} - \sum_{k \in S} \hat{f}_k(X_k^{(i)}) + \epsilon_i - \bar{\epsilon}
\end{align*}
Where $\bar{f^*} = \frac{1}{n} \sum_{i=1}^n f^*(X_S^{(i)})$ and likewise for $\bar{\epsilon}$.

$\epsilon_i$ is subgaussian with subgaussian norm $\sigma$. For a single $\epsilon_i$, we have that $P( |\epsilon_i| \geq t ) \leq C \exp( - c \frac{1}{\sigma^2} t^2)$. Therefore, with probability at least $1-\delta$, $|\epsilon_i| \leq \sigma\sqrt{ \frac{1}{c} \log \frac{C}{\delta}}$. 

By union bound, with probability at least $1-\delta$, $\max_i |\epsilon_i| \leq \sigma \sqrt{ \frac{1}{c} \log \frac{2nC}{\delta}}$.

Also, $|\bar{\epsilon}| \leq \sigma \sqrt{\frac{c}{n} \log \frac{C}{\delta}}$ with probability at least $1 - \delta$.

We know that $|f^*(x)| \leq B$ and $|\hat{f}_k(x_k)| \leq B$ for all $k$. 

Then $|\bar{f^*}| \leq B$ as well, and $|f^*(X_S^{(i)}) - \bar{f^*} - \sum_{k \in S} \hat{f}_k(X_k^{(i)})| \leq 3s B$.

Therefore, taking an union bound, we have that with probability at least $1-\frac{C}{n}$, 
\[
\| \hat{r} \|_\infty \leq (3 s B + c\sigma\sqrt{\log n}) 
\]


\textbf{Step 2.} We now bound $\frac{1}{n} \hat{r}^\tran \max( X, X_{(j)} \mathbf{1})$.

\[
\frac{1}{n} \hat{r}^\tran \max(X, X_{(j)} \mathbf{1}) = \frac{1}{n} \sum_{i=1}^n \hat{r}_i \max(X_i, X_{(j)}) = \frac{1}{n} \sum_{i=1}^n \hat{r}_i X_i \delta(ord(i) \geq j) + \frac{1}{n} X_{(j)} \mathbf{1}_A^\tran \hat{r}_A
\]

Where $A = \{ i : ord(i) \geq j\}$ and $ord(i)$ is the order of sample $i$ where $(1)$ is the smallest element.

We will bound both terms.

\textbf{Term 1.}
\[
\trm{Want to bound} \qquad F(X_1,...,X_n) \coloneqq \frac{1}{n}\sum_{i=1}^n \hat{r}_i X_i \delta(ord(i) \geq j) 
\]

First, we note that $X_i$ is bounded in the range $[-b,b]$. 

We claim then that $F$ is coordinatewise-Lipschitz. Let $X = (X_1, X_2,...,X_n)$ and $X' = (X'_1, X_2, ..., X_n)$ differ only on the first coordinate. 

The order of coordinate $i$ in $X$ and $X'$ can change by at most $1$ for $i \neq 1$. Therefore, of the $j-1$ terms of the series, at most 2 terms differ from $F(X)$ to $F(X')$. Therefore,

$$ | F(X_1, ..., X_n) - F(X'_1, ..., X_n) | \leq \frac{4 b \|\hat{r}\|_\infty}{n} $$

By McDiarmid's inequality therefore, 
\[
P( | F(X) - \E F(X) | \geq t ) \leq C \exp( - c n\frac{t^2}{ (4 b \|\hat{r}\|_\infty)^2})
\]

By symmetry and the fact that $\hat{r}$ is centered, $\E F(X) = 0$.

We can fold the 4 into the constant $c$. With probability $1-\delta$, $|F(X)| \leq b \|\hat{r}\|_\infty \sqrt{   \frac{1}{cn} \log \frac{C}{\delta}}$. 

\textbf{Term 2:}

\[
\trm{Want to bound}   \frac{1}{n} X_{(j)} \mathbf{1}_A^\tran \hat{r}_A
\]

$A$ is a random set and is probabilistically independent of $\hat{r}$.  $\mathbf{1}_A^\tran \hat{r}_A$ is the sum of a sample of $\hat{r}$ without replacement. Therefore, according to Serfling's theorem (Corollary~\ref{cor:serfling}), with probability at least $1-\delta$, $|\frac{1}{n} \mathbf{1}_A^\tran \hat{r}_A|$ is at most $\| \hat{r} \|_\infty \sqrt{ \frac{1}{cn} \log \frac{C}{\delta}}$.

Since $|X_{(j)}|$ is at most $b$, we obtain that with probability at least $1-\delta$, $| \frac{1}{n} X_{(j)} \mathbf{1}_A^\tran \hat{r}_A| \leq b\| \hat{r} \|_\infty \sqrt{ \frac{1}{cn} \log \frac{C}{\delta}}$.

Now we put everything together.

Taking union bound across $p$ and $n$, we have that with probability at least $1-\delta$, 

\[
| \frac{1}{n} \max(X, X_{(j)}\mathbf{1})^\tran \hat{r}| \leq b \|\hat{r}\|_\infty \sqrt{ \frac{1}{c} \frac{1}{n} \log \frac{np C }{\delta}}
\]

Taking union bound and substituting in the probabilistic bound on $\|\hat{r}\|_\infty$, we get that with probability at least $1-\frac{C}{n}$,

$| \frac{1}{n} \max(X, X_{(j)}\mathbf{1})^\tran \hat{r}|$ is at most 
\[
 c b (sB + \sigma) \sqrt{ \frac{s}{n} \log n \log (pn) }
\]

\end{proof}

 
 
 
 
 
 \subsection{Proof of False Negative Control}
 \label{sec:false_negative_proof}
 
 We will use covering number and uniform convergence and will thus need to first introduce some notations. 
 
 Given samples $X^{(1)},...,X^{(n)}$, let $f, g$ be a function and $w$ be a $n$-dimensional random vector, then we denote $\| f - g + w \|_n^2 \coloneqq \frac{1}{n} \sum_{i=1}^n ( f(X^{(i)}) - g(X^{(i)}) + w_i )^2$.

For a function $g : \R^s \rightarrow \R$, define $\hat{R}_s(g) \coloneqq \| f_0 + w - g \|_n^2$ as the objective of the \emph{restricted} regression and define $R_s(g) \coloneqq \E | f_0(X)  + w - g(X) |^2$ as the population risk.

For an additive function $g$, define $\rho_n(g) = \sum_{k=1}^s \| \partial g_k \|_\infty$. Because we always use the secant linear piece-wise function in our optimization program, we define $\ds \| \partial g_k\|_\infty \coloneqq \max_{i=1,...,n-1} |\frac{ g_k(X^{(i)}) - g_k(X^{(i+1)})}{ X^{(i)} - X^{(i+1)}}|$.

Let $\mathcal{C}[b,B,L]$ be the set of 1 dimensional convex functions on $[-b,b]$ that are bounded by $B$ and $L$-Lipschitz.

Let $\mathcal{C}^s[b,B,L]$ be the set of additive functions with $s$ components each of which is in $\mathcal{C}[b,B,L]$. 
\[
\mathcal{C}^s[b,B,L] \coloneqq \{ f : \R^s \rightarrow \R \,:\, f = \sum_{k=1}^s f_k(x_k), \, f_k \in \mathcal{C}[b,B,L] \}
\]
 

We now restate the theorem in our newly defined notation.
 
\begin{theorem} 
Suppose assumptions A1', A2', A3 hold. Suppose $f_0$ depends on all $s$-variables. 

Let $\ds \hat{f} \coloneqq \arg\min \{ \hat{R}_s(f) + \lambda_n \rho_n(f) \,:\, f \in \mathcal{C}^s[b,B,L], f_k \trm{ centered} \}$.

Suppose $n$ is large enough such that $c L \max \left(\lambda_n, b (B+\sigma)B\sigma \sqrt{\frac{1}{n^{4/5}} s^5 \log sn} \right) \rightarrow 0$.

Then, with probability at least $1-\frac{C}{n}$, $\hat{f}_k \neq 0$ for all $k =1,...,s$.
\end{theorem}

\begin{proof}
Let us first sketch out the rough idea of the proof. We know that in the population setting, the best approximate additive function $f^{*s}$ has $s$ non-zero components. We also know that the empirical risk approaches the population risk uniformly. Therefore, it cannot be that the empirical risk minimizer maintains a zero component for all $n$; if that were true, then we can construct a feasible solution to the empirical risk optimization, based on $f^{*s}$, that achieves lower empirical risk. 

Define $f^{*s} = \arg\min \{ R_s(f) \,|\, f \in \mathcal{C}^s[b,B,L], \E f_k(X_k) = 0\}$.

Define $f^{*(s-1)} = \arg\min \{ R_s(f) \,|\, f \in \mathcal{C}^{(s-1)}[b,B,L], \E f_k(X_k) = 0\}$, the optimal solution with only $s-1$ components.

By Theorem~\ref{thm:convex_faithful}, $R_s(f^*_s) - R_s(f^*_{(s-1)}) \geq \alpha > 0$.

$f^{*s}$ is not directly a feasible solution to the empirical risk minimization program because it is not empirically centered. Given $n$ samples, $f^{*s} - \bar{f}^{*s}$ is a feasible solution where $\bar{f}^{*s} = \sum_{k=1}^s \bar{f}_k^{*s}$ and $\bar{f}_k^{*s} = \frac{1}{n} \sum_{i=1}^n f_k^{*s}(X^{(i)})$. 

\begin{align*}
|\hat{R}_s(f^{*s} - \bar{f}^{*s}) - \hat{R}_s(f^{*s})| &\leq \| y - f^{*s} + \bar{f}^{*s} \|_n^2  - \| y - f^{*s}\|_n^2 \\
	&\leq 2\|y-f^{*s}\|_n \|\bar{f}^{*s}\|_n + \|\bar{f}^{*s}\|_n^2 
\end{align*}

Because each $f_k^{*s}$ is bounded by $B$, by Hoeffding inequality, with probability at least $1-\frac{C}{n}$, $|\bar{f}_k^{*s}| \leq B\sqrt{\frac{1}{cn} \log n}$. By an union bound therefore, with probability at least $1-\frac{C}{n}$, $\|\bar{f}^{*s}\|_n \leq B\sqrt{\frac{1}{cn} \log sn}$.

\begin{align*}
\| y - f^{*s} \|_n &= \| f_0 + w - f^{*s} \|_n \\
	&\leq \| f_0 - f^{*s} \|_n + \|w\|_n \\
\end{align*}

$f_0 - f^{*s}$ is bounded by $2sB$ and $w_i$ is zero-mean subgaussian with scale $\sigma$. Therefore, $\|w\|_n$ is at most $c\sigma$ with probability at least $1-\frac{C}{n}$ for all $n > n_0$. 

So we derive that, with probability at least $1 -\frac{C}{n}$, for all $n > n_0$,
\[
|\hat{R}_s(f^{*s} - \bar{f}^{*s}) - \hat{R}_s(f^{*s})|\leq 2csB(B+\sigma) \sqrt{\frac{1}{cn} \log sn}
\]

Suppose $\hat{f}$ has at most $s-1$ non-zero components. Then
\begin{align*}
\hat{R}_s( \hat{f}) &\geq R_s(\hat{f}) - \tau_n \\
	&\geq R_s(f^{*(s-1)}) - \tau_n \\
	&\geq R_s(f^{*s}) + \alpha - \tau_n \\
	&\geq \hat{R}_s(f^{*s}) + \alpha - 2 \tau_n \\
	&\geq \hat{R}_s(f^{*s} - \bar{f}^{*s}) -\tau'_n + \alpha - 2\tau_n 
\end{align*}
Where $\tau_n$ is the deviation between empirical risk and true risk and $\tau'_n$ is the approximation error incurred by empirically sampling $f^{*s}$.

Adding and subtracting $ \lambda_n \rho_n(f^{*s} - \bar{f}^{*s})$ and $ \lambda_n \rho_n(\hat{f})$, we arrive at the conclusion that
\[
\hat{R}_s(\hat{f}) + \lambda_n \rho_n(\hat{f}) \geq \hat{R}_s(f^{*s} - \bar{f}^{*s}) + \lambda_n \rho_n(f^{*s} - \bar{f}^{*s}) - ( \lambda_n \rho_n(f^{*s} - \bar{f}^{*s}) + \lambda_n \rho_n(\hat{f}) ) -\tau'_n + \alpha - 2 \tau_n
\]

$\rho_n(\hat{f}), \rho_n(f^{*s} - \bar{f}^{*s})$ are at most $L$. By Theorem~\ref{thm:uniform_risk_deviation}, we know that under the condition of the theorem, $\tau_n \leq  bLB\sigma(B+\sigma)\sqrt{ \frac{1}{cn^{4/5}} s^5 \log n}$.

$|\lambda_n \rho_n(\hat{f}) - \lambda_n \rho_n(f^*_s)| \leq 2L \lambda_n$.

$\tau'_n$, as shown above, is at most $2sB(B+\sigma) \sqrt{\frac{1}{cn} \log sn}$ with probability at least $1-\frac{C}{n}$ for $n > n_0$.

For $n$ large enough such that
\[
c\max(L\lambda_n,  bLB\sigma(B+\sigma) \sqrt{ \frac{1}{n^{4/5}} s^5 \log sn}) < \alpha
\]
we get that $\hat{R}_s(\hat{f}) + \lambda_n \rho_n(\hat{r}) > \hat{R}_s(f^*_s) + \lambda_n \rho_n(f^*_s)$, which is a contradiction.

\end{proof}

%

\begin{theorem} (Uniform Risk Deviation)
\label{thm:uniform_risk_deviation}
For all $n > n_0$, we have that, with probability at least $1 - \frac{C}{n}$, 
\[
\sup_{f \in \mathcal{C}^s[b,B,L]} |\hat{R}_s(f) - R_s(f)| \leq B\sigma(B+\sigma)Lb\sqrt{ \frac{1}{cn^{4/5}} s^5 \log sn}
\]
\end{theorem}


\begin{proof}

Let $\mathcal{C}_0^{s}[b,B,L]$ be an $\epsilon$-cover of $\mathcal{C}^s[b,B,L]$. 

For all $f \in \mathcal{C}^s[b,B,L]$,
\[
\hat{R}_s(f) - R_s(f) = \hat{R}_s(f) - \hat{R}_s(f') + \hat{R}_s(f') - R_s(f') + R_s(f') - R_s(f)
\]
where $f' \in \mathcal{C}_0^s[b,B,L]$ and $\| f - f' \|_\infty \leq \epsilon$.

We first bound $\hat{R}_s(f) - \hat{R}_s(f')$.

\begin{align*}
|\hat{R}_s(f) - \hat{R}_s(f')| &= |\| f_0 + w - f \|_n^2 - \| f_0 + w - f' \|_n^2| \\
	&\leq 2 \langle f_0 + w , f' - f \rangle_n + \| f \|_n^2 - \| f' \|_n^2\\
	&\leq 2 \| f_0 + w \|_n \| f' - f \|_n + ( \| f \|_n - \|f'\|_n)( \| f \|_n + \|f'\|_n)\\
\end{align*}

$\| f_0 + w \|_n \leq \| f_0 \|_n + \|w \|_n$. $\| w \|^2_n = \frac{1}{n}\sum_{i=1}^n w_i^2$ is the average of subexponential random variables. Therefore, for all $n$ larger than some absolute constant $n_0$, with probability at least $1-\frac{C}{n}$, $|\|w\|^2_n - \E|w|^2 | < \sigma^2 \sqrt{\frac{1}{cn} \log n}$. The absolute constant $n_0$ is determined so that for all $n > n_0$, $\sqrt{\frac{1}{cn} \log n} < 1$. 

$\|f_0\|^2_n$ is the average of random variables bounded by $B^2$ and therefore, with probability at least $1-\frac{C}{n}$, $|\| f_0 \|^2_n - \E |f_0(X)|^2 | \leq B^2\sqrt{ \frac{1}{cn} \log n}$.

Since $\E|w|^2 \leq c\sigma^2$ and $\E| f_0(X)|^2 \leq B^2$, we have that for all $n \geq n_0$, with probability at least $1-\frac{C}{n}$, $\| f_0 + w \|_n \leq c(B+\sigma)$.

$\| f' - f \|_\infty \leq \epsilon$ implies that $\| f' - f\|_n \leq \epsilon$. And therefore, $\| f \|_n - \|f'\|_n \leq \| f - f' \|_n \leq \epsilon$. 

$f,f'$ are all bounded by $sB$, and so $\|f \|_n, \| f' \|_n \leq sB$.

Thus, we have that, for all $n > n_0$,
\begin{align}
|\hat{R}_s(f) - \hat{R}_s(f')| \leq \epsilon cs(B + \sigma) \label{eqn:approx_deviation1}
\end{align}
with probability at least $1-\frac{C}{n}$.

Now we bound $R_s(f') - R_s(f)$. The steps follow the bounds before, and we have that 
\begin{align}
|R_s(f') - R_s(f)| \leq \epsilon cs(B + \sigma) \label{eqn:approx_deviation2}
\end{align}

Lastly, we bound $\sup_{f' \in \mathcal{C}_0^s[b,B,L]} \hat{R}_s(f') - R_s(f')$. 

For a fixed $f'$, we have that, by definition
\[
\| f_0 + w - f' \|_n^2 = \| f_0 - f'\|_n^2 + 2 \langle w, f_0 - f'\rangle_n + \| w \|_n^2 \\
\]
Because $f_0(X^{(i)}) - f'^(X^{(i)})$ is bounded by $2sB$, $\| f_0 - f'\|_n^2$ is the empirical average of $n$ random variables bounded by $4(sB)^2$. 

Using Hoeffding Inequality then, we know that the probability $| \| f_0 - f' \|_n^2 - \E(f_0(X) - f '(X))^2 | \geq t $ is at most $C \exp( - c n t^2 \frac{1}{(sB)^4})$. 

Consider now the term $2 \langle w, f_0 - f' \rangle_n \coloneqq \frac{2}{n} \sum_{i=1}^n w_i ( f_0(X^{(i)}) - f'(X^{(i)}))$. We note that $w_i$ and $X^{(i)}$ are independent, $w_i$ is subgaussian. 

The $n$-dimensional vector $\{ \frac{1}{n} ( f_0(X^{(i)}) - f'(X^{(i)})) \}_i$ has norm at most $\frac{sB}{\sqrt{n}}$. Therefore, $|2 \langle w, f_0 - f' \rangle_n | \geq t$ with probability at most $C \exp( - c n t^2 \frac{1}{\sigma^2(sB)^2})$. 

The last term $\| w \|_n^2 = \frac{1}{n} \sum_{i=1}^n w_i^2$. Using subexponential concentration, we know that $ |\| w \|_n^2 - \E|w|^2 | \geq t$ occurs with probability at most $C \exp( - c n \frac{1}{\sigma^2} )$ for $n$ larger than some $n_0$.

Collecting all these results and applying union bound, we have that $\sup_{f' \in \mathcal{C}_0^s[b,B,L]} | \hat{R}_s(f') - R_s(f') | \geq t$ occurs with probability at most
\[
C \exp( s\left( \frac{bBLs}{\epsilon} \right)^{1/2} - c n t^2 \frac{1}{\sigma^2 (sB)^4} )
\]
for all $n > n_0$.

Restating, we have that with probability at most $1- \frac{1}{n}$, the deviation is at most 
\begin{align}
\sqrt{ \frac{1}{cn} \sigma^2 (sB)^4 \left( \log Cn + s( \frac{bBLs}{\epsilon})^{1/2} \right)} \label{eqn:finite_cover_deviation}
\end{align}
Substituting in $\epsilon = \frac{bBLs}{n^{2/5}}$, expression~\ref{eqn:finite_cover_deviation} becomes $\sqrt{ \frac{1}{cn^{4/5}} \sigma^2 s^5 B^4 \log Cn}$.

Expressions~\ref{eqn:approx_deviation1} and \ref{eqn:approx_deviation2} become $\sqrt{ \frac{(bBLs)^2}{cn^{4/5}}} (B+\sigma)$.

\end{proof}
 
 
 
 
 
 
 
 \subsection{Supporting Technical Material}
 
 \subsubsection{Concentration of Measure}

\textbf{Sub-Exponential} random variable is the square of a subgaussian random variable\cite{vershynin2010introduction}.

\begin{proposition} (Subexponential Concentration \cite{vershynin2010introduction})
Let $X_1,...,X_N$ be zero-mean independent subexponential random variables with subexponential scale $K$. 
\[
P( | \frac{1}{N} \sum_{i=1}^N X_i | \geq \epsilon) \leq
	2 \exp \left[ -c N \min\left( \frac{\epsilon^2}{K^2}, \frac{\epsilon}{K} \right) \right]
\]
where $c > 0$ is an absolute constant.
\end{proposition}

For uncentered subexponential random variables, we can use the following fact. If $X_i$ subexponential with scale $K$, then $X_i - \E[X_i]$ is also subexponential with scale at most $2K$.

\textbf{Restating}. We can set
\[
c \min\left( \frac{\epsilon^2}{K^2}, \frac{\epsilon}{K} \right) = \frac{1}{N} \log \frac{1}{\delta}.
\]
Thus, with probability at least $1-\delta$, the deviation at most
\[
K \max\left( \sqrt{\frac{1}{cn} \log \frac{C}{\delta}},  \frac{1}{cn} \log \frac{C}{\delta} \right)
\]

\begin{corollary}
Let $w_1,...,w_n$ be $n$ independent subgaussian random variables with subgaussian scale $\sigma$. 

Then, for all $n > n_0$, with probability at least $1- \frac{1}{n}$,
\[
\frac{1}{n} \sum_{i=1}^n w_i^2 \leq c \sigma^2 
\]
\end{corollary}

\begin{proof}
Using the subexponential concentration inequality, we know that, with probability at least $1-\frac{1}{n}$, 

\[
| \frac{1}{n} \sum_{i=1}^n w_i^2 - \E w^2 | \leq \sigma^2 \max\left( \sqrt{\frac{1}{cn} \log \frac{C}{\delta}}, \frac{1}{cn}\log \frac{C}{\delta} \right)
\]

First, let $\delta = \frac{1}{n}$. Suppose $n$ is large enough such that $ \frac{1}{cn} \log Cn < 1$. Then, we have, with probability at least $1-\frac{1}{n}$,
\begin{align*}
 \frac{1}{n} \sum_{i=1}^n w_i^2 &\leq c\sigma^2 (1+\sqrt{\frac{1}{cn} \log Cn}) \\
		&\leq 2 c \sigma^2
 \end{align*}
 
\end{proof}

\subsubsection{Sampling Without Replacement}

\begin{lemma} (Serfling \cite{serfling1974probability}) 
Let $x_1,..., x_N$ be a finite list, $\bar{x} = \mu$. Let $X_1,...,X_n$ be sampled from $x$ without replacement. 

Let $b = \max_i x_i$ and $a = \min_i x_i$. Let $r_n = 1- \frac{n-1}{N}$. Let $S_n = \sum_i X_i$.
Then we have that
\[
P( S_n - n \mu \geq n \epsilon) \leq \exp( - 2 n \epsilon^2 \frac{1}{r_n (b-a)^2})
\]
\end{lemma}

\begin{corollary}
\label{cor:serfling}
Suppose $\mu = 0$. 
\[
P( \frac{1}{N} S_n \geq \epsilon) \leq \exp( -2 N \epsilon^2 \frac{1}{(b-a)^2})
\]

And, by union bound, we have that
\[
P( | \frac{1}{N} S_n| \geq \epsilon) \leq 2 \exp( -2 N \epsilon^2 \frac{1}{(b-a)^2})
\]

\end{corollary}

A simple restatement. With probability at least $1- \delta$, the deviation $| \frac{1}{N} S_n$ is at most $ (b-a) \sqrt{ \frac{1}{2N} \log \frac{2}{\delta}}$.s

\begin{proof}
\[
P( \frac{1}{N} S_n \geq \epsilon) = P( S_n \geq \frac{N}{n} n \epsilon) \leq \exp( - 2 n \frac{N^2}{n^2} \epsilon^2 \frac{1}{r_n (b-a)^2} ) 
\]

We note that $r_n \leq 1$ always, and $n \leq N$ always. 
\[
\exp( - 2 n \frac{N^2}{n^2} \epsilon^2 \frac{1}{r_n (b-a)^2} )  \leq \exp( - 2 N \epsilon^2 \frac{1}{(b-a)^2})
\]
This completes the proof.

\end{proof}

\subsubsection{Covering Number for Lipschitz Convex Functions}

\begin{definition}
$\{ f_1,..., f_N\} \subset \mathcal{C}[b,B,L]$ is an $\epsilon$-covering of $\mathcal{C}[b,B,L]$ if for all $f \in \mathcal{C}[b,B,L]$, there exist $f_i$ such that $\| f - f_i \|_\infty \leq \epsilon$.

We define $N_\infty( \epsilon, \mathcal{C}[b,B,L])$ as the size of the minimum covering.
\end{definition}

\begin{lemma} (Bronshtein 1974)
\[
\log N_\infty (\epsilon, \mathcal{C}[b,B,L]) \leq C\left( \frac{bBL}{\epsilon} \right)^{1/2}
\]
For some absolute constant $C$.
\end{lemma}

\begin{lemma}
\[
\log N_\infty( \epsilon, \mathcal{C}^s[b,B,L])  \leq C s \left(\frac{bBLs}{\epsilon}\right)^{1/2}
\]
For some absolute constant $C$.
\end{lemma}

\begin{proof}
Let $f = \sum_{k=1}^s f_k$ be a convex additive function. Let $\{ f'_k \}_{k=1,..,s}$ be $k$ functions from a $\frac{\epsilon}{s}$ $L_\infty$ covering of $\mathcal{C}[b,B,L]$. 

Let $f' \coloneqq = \sum_{k=1}^s f'_k$, then 
\[
\| f' - f \|_{\infty} \leq \sum_{k=1}^s \| f_k - f'_k \|_\infty \leq s \frac{\epsilon}{s} \leq \epsilon
\]

Therefore, a product of $s$ $\frac{\epsilon}{s}$-coverings of univariate functions induces an $\epsilon$-covering of the additive functions.
\end{proof}

 
