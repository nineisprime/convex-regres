\section{Introduction}


We consider the problem of estimating a convex function of several
variables from noisy values of the function at a finite sample of
input points.  Recent work \cite{Guntu:12,Guntu:13} shows that the
minimax rate for convex function estimation in $p$ dimensions is
$n^{-4/(4+p)}$.  Loosely speaking, this shows that the
geometric convexity constraint is statistically equivalent to
requiring two derivatives of the function, and thus is
subject to the same curse of dimensionality.
However, if the function is sparse, with $s\ll p$ relevant variables,
then the faster rate $n^{-4/(4+s)}$ may be achievable if the $s$
variables can be identified.  To determine the relevant variables, we
show that it suffices to estimate a sum of $p$ one-dimensional convex
functions, leading to significant computational and statistical
advantages.  In addition, we introduce algorithms and supporting
statistical theory for a practical, effective approach to this
variable selection problem.

The general sparse nonparametric regression problem is considered in \cite{lafferty2008rodeo}, where it is shown that computationally efficient, near minimax-optimal estimation is possible, but in ambient
dimensions that scale only as $p = O(\log n)$ instead of $p=O\bigl(e^{n^c}\bigr)$ as enjoyed by sparse linear models. \citet{dalalyan:12} do achieve exponential scaling $p=O(e^n)$ under certain Fourier smoothness conditions but they show that variable selection is still hard in that the number of relevant variables $s$ must be less than $\log n$.

% The general sparse nonparametric regression problem is considered in
% \cite{lafferty2008rodeo}, where it is shown that computationally
% efficient, near minimax-optimal estimation is possible, but in ambient
% dimensions that scale only as $p = O(\log n)$.  This is in stark
% contrast to the exponential scaling $p = O\bigl(e^{n^c}\bigr)$ enjoyed
% by sparse linear models \cite{Wain:09a}.  

Approximating the regression function by a sum of one-dimensional functions, known as sparse additive models, \cite{Ravikumar:09} is a practical alternative to fully
nonparametric function estimation.  But the additive assumption is
limited.  In particular, the natural idea of first selecting the
single variable effects, then the pairwise effects, and so on, does
not in general lead to consistent variable selection.  In other words,
the general nonparametric model is not additively faithful.
Remarkably, the additional assumption of convexity does lead to
consistent variable selection, as we show here. In addition, we show that
the scaling $p = O(\log n)$ and $n = O\big(\textrm{poly}(s)\big)$ is achievable for sparse convex additive models. Thus, the geometric convexity constraint is quite different from the
smoothness constraints imposed in traditional nonparametric
regression.

A key to our approach is the observation that least squares
nonparametric estimation under convexity constraints is equivalent to
a finite dimensional quadratic program.  Specifically, the infinite
dimensional optimization 
\begin{align}
\begin{split}
\text{minimize} & \quad \sum_{i=1}^n (Y_i - m(x_i))^2 \\
\text{subject to} &  \quad m:\reals^p\rightarrow\reals\ \text{is
  convex}
\end{split}
\end{align}
is precisely equivalent to the finite dimensional quadratic
program 
\begin{align}
\begin{split}
\label{eq:outer}
\text{minimize}_{h, \beta} & \;\; \sum_{i=1}^n (Y_i - h_i)^2 \\
\text{subject to} & \;\; h_j \geq h_i + \beta_i^T (x_j-x_i),\; \text{for
    all $i,j$}.
\end{split}
\end{align}
%\end{equation}
%See \cite{Boyd04}, Section 6.5.5.
Here $h_i$ is the estimated function value $m(x_i)$, and the vectors
$\beta_i \in \reals^d$ represent supporting hyperplanes to the
epigraph of $m$.  Importantly, this finite dimensional quadratic program does
not have tuning parameters for smoothing the function. Such parameters are the bane
of nonparametric estimation.

Estimation of convex functions arises naturally in several
applications.  Examples include geometric programming \cite{Boyd04},
computed tomography \cite{Prince:90}, target reconstruction
\cite{Lele:92}, image analysis \cite{Golden:06} and circuit design
\cite{Hannah:12}.  Other applications include queuing theory
\cite{Chen:01} and economics, where it is of interest to estimate
concave utility functions \cite{Pratt:68}.  See \cite{Lim:12} for
other applications.  

Beyond cases where the assumption of convexity is
natural, we offer that the convexity assumption is attractive as a
tractable, nonparamametric relaxation of the linear model.  In
addition to the lack of tuning parameters, other than the
regularization parameter $\lambda$ to control the level of sparsity,
the global convexity assumption leads to effective, scalable algorithms.  We
demonstrate use of our approach on experiments with standard
regression data sets, in a comparison with sparse linear models
(lasso).

%In the following section we give a high-level summary of our technical
%results, including additive faithfulness, variable selection 
%consistency, and high dimensional scaling.  In Section~X...

\def\mathbf#1{\mbox{\boldmath $#1$}} 

