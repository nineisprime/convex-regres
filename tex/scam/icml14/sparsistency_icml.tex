\section{Analysis of Variable Selection Consistency}

We divide our analysis into two parts. We first establish a sufficient
\emph{deterministic} condition for sparsistency.  We then consider the
stochastic setting and argue that the deterministic conditions hold with high probability. 

\subsection{Deterministic Setting}

We follow \cite{Wain:09a} and define the \emph{restricted regression} purely for theoretical purposes.
\begin{definition}
In \emph{restricted regression}, we restrict the indices $k$ in
optimization (\ref{opt:alternate_opt}) to lie in the support $S$ instead of ranging from $1,...,p$. 
\end{definition}

Our analysis then differs from the now-standard ``primal-dual witness
technique''~\cite{Wain:09a}. Primal-dual witness explicitly solves all the dual variables, but because our optimization is more complex, we do not solve the dual variables on $S$; we instead write the dual variables on $S^c$ as a function of the restricted regression \emph{residual}, which is implicitly a function of the dual variables on $S$.

\begin{theorem} (Deterministic setting)
\label{thm:deterministic}
Let $\{\hat{d}_k, \hat{c}_k\}_{k \in S}$ be the minimizer of the restricted regression, that is, the solution to optimization (\ref{opt:alternate_opt}) where we restrict $k \in S$. Let $\hat{d}_k = 0$ and $ \hat{c}_k = 0$ for $k \in S^c$.
Let $\hat{r} \coloneqq Y - \bar{Y}\mathbf{1}_n - \sum_{k \in S} (\Delta_k \hat{d}_k -
\hat{c}_k \mathbf{1}_n)$ be the restricted regression residual. For $k
\in \{1,...,p\}$, Let $\Delta_{k, j} \in R^n$ be the $j$-th column of $\Delta_k$, i.e. $\max( X_k - X_{k (j)} \mathbf{1}_n, 0)$. \\

Suppose for all $j$ and all $k\in S^c$, $\lambda_n > | \frac{1}{n}
\hat{r}^\tran \Delta_{k,j}|$. Then $\hat{\mu}$ and $\hat{d}_k, \hat{c}_k$ for $k=1,...,p$ is an optimal solution to the full regression \ref{opt:alternate_opt}. Furthermore, any solution to the optimization program \ref{opt:alternate_opt} must be zero on $S^c$.
\end{theorem}

This result holds regardless of whether we impose the boundedness and Lipschitz conditions in optimization~\ref{opt:alternate_opt}.
The full proof of Theorem~\ref{thm:deterministic} is in Section~\ref{sec:deterministic_proof} of the Appendix.

\begin{remark}
  The incoherence condition of \cite{Wain:09a} is implicitly encoded
  in our condition on $\lambda_n, \hat{r}, \Delta_{k,j}$. We can
  reconstruct the incoherence condition if we assume that the true
  function $f_0$ is linear and that our fitted functions $\hat{f}_k$
  are linear as well.
\end{remark}

Theorem~\ref{thm:deterministic} allows us to analyze false negative
rates and false positive rates separately. To control false positives,
we study when the condition $\lambda_n > | \frac{1}{n} \hat{r}^\tran
\Delta_{k,j}|$ is fulfilled for all $j$ and all $k \in S^c$. To
control false negatives, we study the restricted regression.

\subsection{Probabilistic Setting}

We use the following statistical setting:

\begin{packed_enum}
\item Let $F$ be a distribution supported and positive on $\mathcal{X}=[-b,b]^p$. Let $X^{(1)},..., X^{(n)} \sim F$ be iid.
\item Let $Y = f_0(X) + \epsilon$ where $\epsilon$ is zero-mean noise. Let $Y^{(1)},...,Y^{(n)}$ be iid.
\item Let $S = \{1,...,s\}$ denote the relevant variables where $s\leq p$, i.e.,
  $f_0(X) = f_0(X_S)$.
\item Let $f^*_1,...,f^*_s \coloneqq \argmin_{f_1,...,f_s} \{ \E\Big(f_0(X) -\E f_0(X) - \sum_{k=1}^s f_k(X_k)\Big)^2 \,|\, \E[f_k(X_k)] = 0 \}$.
\end{packed_enum}

Each of our theorems will use a subset of the following assumptions:
\begin{packed_enum}
\item[A1:] $X_S, X_{S^c}$ are independent.  \ A1': $\{ X_k \}_{k \in S}$ are independent.
\item[A2:] $\|f_0\|_\infty \leq sB$ \  A2': $f_0$ is convex,
  twice-differentiable, and $L$-Lipschitz.
\item[A3:] Suppose $\epsilon$ is mean-zero sub-Gaussian, independent of $X$, with sub-Gaussian scale $\sigma$, i.e. for all $t \in \R$, $\E e^{t \epsilon} \leq e^{\sigma^2 t^2 / 2}$.
\item[A4:] For all $k=1,...,s$, $\E(f^*_s(X_k))^2 \geq \alpha$ for some positive constant $\alpha$.
\end{packed_enum}

We will use assumptions A1, A2, A3 to control the probability of false
positives and the stronger assumptions A1', A2', A3, A4 to control the
probability of false negatives.  Assumption A4 can be
weakened so that the relevant functions satisfy
$\E(f^*_s(X_k))^2 \geq \alpha_n$ for $\alpha_n$ decaying to zero 
at an appropriate rate.



\begin{remark}
  We make strong assumptions on the covariates in A1 in order to make
  very weak assumptions on the true regression function $f_0$ in
  A2. In particular, we do not assume that $f_0$ is additive. Relaxing
  these assumptions is an interesting direction for future work. 
  %Strong assumptions on the covariates are not uncommon in nonparametric
  %variable selection analysis~\cite{lafferty2008rodeo}. 
  %[TODO: refer to correlated design experiment].
\end{remark}

\begin{remark}
Assumption A4 ensures that the relevant variables are ``relevant enough''. Under A4, the population risk of an additive function with $s-1$ components is at least $\alpha$ larger than the population risk of the optimal additive function with $s$ components. See lemma~\ref{lem:minus_one_risk_increase} in section~\ref{sec:false_negative_proof} of the appendix.
\end{remark}

\begin{theorem} (Controlling false positives) 
\label{thm:false_positive}
Suppose assumptions A1, A2, A3 hold. Suppose also that we run optimization~\eqref{opt:alternate_opt} with the $B$-boundness constraint. Let $c,C$ be absolute constants.
Suppose $\lambda_n \geq c b (sB + \sigma) \sqrt{ \frac{s}{n} \log n
  \log (pn)}$.  Then with probability at least $ 1 - \frac{C}{n}$, for all $j,k$, $\lambda_n >  | \frac{1}{n} \hat{r}^\tran \Delta_{k,j}|$.
Therefore, any solution to the full regression (\ref{opt:alternate_opt}), with boundedness constraint, is zero on $S^c$. 
\end{theorem}

The proof of Theorem~\ref{thm:false_positive} exploits independence of
$\hat{r}$ and $\Delta_{k,j}$ from A1, and then uses concentration of
measure results to argue that $| \frac{1}{n} \hat{r}^\tran
\Delta_{k,j}|$ concentrates around zero at a desired rate. The fact
that $\hat{r}$ is a centered vector is crucial to our proof, and our
theory thus further illustrates the importance of imposing the
centering constraints in optimization \eqref{opt:alternate_opt}. Our
proof uses the concentration of the average of
data sampled \emph{without} replacement
\cite{serfling1974probability}, illustrating that the proof method is not a
trivial application of existing techniques. The full proof of
Theorem~\ref{thm:false_positive} is in
Section~\ref{sec:false_positive_proof} of the Appendix.

\begin{theorem} (Controlling false negatives)
\label{thm:false_negative}
Suppose assumptions A1', A2', A3, A4 hold. Let $\ds \hat{f} = \{ \hat{d}_k, \hat{c}_k\}_{k\in S}$ be any solution to the restricted regression with both the $B$-boundedness and $L$-Lipschitz constraint. Let $c,C$ be absolute constants.
Suppose $s L \lambda_n \rightarrow 0$ and $Lb(sB+\sigma)sB \sqrt{\frac{s}{n^{4/5}} \log sn} \rightarrow 0$.
Then, for sufficiently large $n$, $\hat{f}_k = (\hat{d}_k, \hat{c}_k)
\neq 0$ for all $k \in S$ with probability at least $1-\frac{C}{n}$.
\end{theorem}

This is a finite sample version of
Theorem~\ref{thm:convex_faithful}. We need stronger assumptions in
Theorem~\ref{thm:false_negative} to use our additive faithfulness
result, Theorem~\ref{thm:convex_faithful}. We also include an extra
Lipschitz constraint so that we can use existing covering number
results \cite{Bronshtein:76}. Recent work
\cite{Guntu:13} shows that the Lipschitz constraint
is not required with more advanced empirical process theory
techniques. We give the full proof of Theorem~\ref{thm:false_negative}
in Section~\ref{sec:false_negative_proof} of the Appendix.

Combining Theorem~\ref{thm:false_positive} and
~\ref{thm:false_negative} and ignoring dependencies on $b,B,L,\sigma$,
we have the following result.
\begin{corollary}
  Assume A1', A2', A3, A4. Let $\lambda_n = \Theta\left( \sqrt{
  \frac{s^3}{n} \log n \log (pn)} \right)$. Suppose $s \lambda_n
  \rightarrow 0$ and $\sqrt{\frac{s^5}{n^{4/5}} \log sn} \rightarrow
  0$. Let $\hat{f_n}$ be a solution to (\ref{opt:alternate_opt}) with
  boundedness and Lipschitz constraints. Then 
  $\P( \trm{supp}(\hat{f_n}) = \trm{supp}(f_0) ) \rightarrow 1$.
\end{corollary}
The above corollary implies that sparsistency is achievable at the same exponential scaling of the ambient dimension $p = O(\exp(n^c)), c<1$ rate as parametric models. The cost of nonparametric modeling is reflected in the scaling with respect to $s$, which can only scale at $o(n^{4/25})$.

\citet{dalalyan:12} have shown that under tradtional smoothness constraints, variable selection is achievable only if $n > O(e^s)$. It is interesting to observe that because of additive faithfulness, the convexity assumption enables a much better scaling of $n = O(\textrm{poly}(s))$, demonstrating that geometric constraints can be quite different from the previously studied smoothness conditions.


%\textbf{Comparison with Related Work.} 


% DO NOT CHANGE; RefTex variables -minx
 
%%% Local Variables: ***
%%% mode:latex ***
%%% TeX-master: "scam_icml.tex" ***
%%% End: ***
